\documentclass{article}
\usepackage[utf8]{inputenc}
\usepackage[T1]{fontenc}
\usepackage{listings}
\usepackage{color}
\usepackage{graphicx}
\usepackage{fontspec}
\usepackage{xurl}
\usepackage[pdftex]{hyperref}
\setlength{\parindent}{0pt}

% YAML code color settings for listings package
\definecolor{gray}{rgb}{0.5,0.5,0.5}
\definecolor{orange}{rgb}{1,0.5,0}
\lstset{
    language=bash,
    basicstyle=\ttfamily,
    keywordstyle=\color{blue},
    commentstyle=\color{gray},
    stringstyle=\color{orange},
    breaklines=true,
    breakatwhitespace=true,
    showstringspaces=false
}


\title{Zarządzanie Systemami Rozproszonymi\\Laboratoria z Microsoft Azure i Terraform}
\author{mgr inż. Jakub Woźniak}
\date{}

\begin{document}

\maketitle
\section{Wprowadzenie do chmury publicznej}

\subsection{Historia Rozwoju: Od Kolokacji do Chmury}
Na początku ery internetu firmy inwestowały we własną infrastrukturę sprzętową, którą umieszczano w centrach danych (kolokacja). Organizacje wynajmowały fizyczne miejsce na serwery, jednak zarządzanie sprzętem i oprogramowaniem pozostawało po ich stronie. Takie rozwiązanie było kosztowne, nieelastyczne i wymagało specjalistycznych zespołów.

Pojawienie się wirtualizacji zrewolucjonizowało rynek i dało początek \textbf{chmurze obliczeniowej}. Pierwszym krokiem był model \textbf{Infrastructure as a Service (IaaS)}. W 2006 roku Amazon uruchomił usługę EC2, umożliwiając firmom wynajem mocy obliczeniowej na żądanie. Wkrótce pojawiły się inne modele:
\begin{itemize}
    \item \textbf{Platform as a Service (PaaS)} – np. Azure App Service, eliminujący konieczność zarządzania systemem operacyjnym.
    \item \textbf{Software as a Service (SaaS)} – np. Microsoft 365, w pełni gotowe oprogramowanie dostępne dla użytkowników.
\end{itemize}

Dziś chmura publiczna, jak \textbf{Microsoft Azure} czy \textbf{AWS}, zapewnia elastyczność, skalowalność i dostęp do zaawansowanych usług przy minimalnych inwestycjach w infrastrukturę.

\subsection{Model Współdzielonej Odpowiedzialności (Shared Responsibility)}
W chmurze publicznej odpowiedzialność za bezpieczeństwo i zarządzanie zasobami jest podzielona między dostawcę usługi a klienta. Jest to tzw. \textbf{Shared Responsibility Model}.

\subsubsection*{Podział odpowiedzialności w zależności od modelu usług}
\begin{itemize}
    \item \textbf{IaaS (Infrastructure as a Service):} Dostawca (np. Microsoft Azure) odpowiada za infrastrukturę fizyczną, ale użytkownik zarządza systemem operacyjnym, aktualizacjami i aplikacjami.
    \item \textbf{PaaS (Platform as a Service):} Dostawca zarządza infrastrukturą i systemem operacyjnym, a użytkownik odpowiada za aplikacje oraz ich konfigurację.
    \item \textbf{SaaS (Software as a Service):} Całość (od sprzętu po aplikacje) jest zarządzana przez dostawcę, a użytkownik odpowiada jedynie za dane i konfigurację kont.
\end{itemize}

\textbf{Przykład:} W usłudze \textit{Azure App Service}, użytkownik implementuje aplikację, a dostawca zajmuje się infrastrukturą i systemem operacyjnym.

\subsection{Demokratyzacja Technologii}
Chmura publiczna odegrała kluczową rolę w \textbf{demokratyzacji technologii}. Usługi takie jak \textbf{Azure Cognitive Services} czy \textbf{AWS Rekognition} umożliwiają dostęp do zaawansowanych funkcji sztucznej inteligencji (AI) dla firm każdej wielkości.

\subsubsection*{Dlaczego to ważne?}
Alternatywą dla chmury były wcześniej drogie, ręcznie implementowane rozwiązania, które wymagały zespołów specjalistów, infrastruktury i dużych inwestycji czasowych. Dzięki chmurze:
\begin{itemize}
    \item Start-up może wykorzystać \textbf{Azure Computer Vision} do analizy obrazu bez trenowania własnych modeli AI.
    \item Firmy płacą tylko za rzeczywiste zużycie zasobów, co radykalnie obniża koszty i próg wejścia.
\end{itemize}

Demokratyzacja technologii przyspiesza innowacje i otwiera rynek dla małych organizacji, które wcześniej nie miały dostępu do zaawansowanych rozwiązań. Dzięki chmurze wystarczy pomysł, aby korzystać z narzędzi na poziomie globalnych korporacji.
\section{Wprowadzenie do Microsoft Azure}
Microsoft Azure to publiczna platforma chmurowa oferująca szeroki zakres usług, takich jak obliczenia, przechowywanie danych, sieci, bazy danych i wiele innych. W tej sekcji studenci poznają podstawowe pojęcia związane z platformą Azure oraz nauczą się korzystać z interfejsu użytkownika, aby stworzyć pierwsze zasoby.

\subsection{Podstawowe pojęcia}
\begin{itemize}
    \item \textbf{Subscription} – Plan subskrypcyjny określający limit zasobów i sposób ich płatności.
    \item \textbf{Tenant} – Logiczna przestrzeń w Microsoft Entra ID (Azure Active Directory), umożliwiająca zarządzanie użytkownikami i zasobami.
    \item \textbf{Resource Group (RG)} – Kontener, który grupuje powiązane zasoby, takie jak maszyny wirtualne, sieci i bazy danych.
\end{itemize}

\subsection{Interfejs użytkownika}
Azure oferuje dwa główne sposoby interakcji:
\begin{itemize}
    \item \textbf{Azure Portal} – Graficzny interfejs użytkownika dostępny w przeglądarce.
    \item \textbf{Azure CLI} – Narzędzie wiersza poleceń umożliwiające automatyzację i skryptowanie.
\end{itemize}

\subsection{Ćwiczenie 1: Tworzenie Resource Group}
\textbf{Cel:} Stworzenie pierwszej Resource Group w Azure Portal.

\begin{enumerate}
    \item Zaloguj się do Azure Portal (\url{https://portal.azure.com}).
    \item Kliknij \texttt{Create a resource}, a następnie wybierz \texttt{Resource Group}.
    \item Wprowadź nazwę grupy, np. \texttt{example-rg}, oraz wybierz region, np. \texttt{West Europe}.
    \item Kliknij \texttt{Review + Create}, a następnie \texttt{Create}.
\end{enumerate}

\subsection{Ćwiczenie 2: Tworzenie maszyny wirtualnej w Resource Group}
\textbf{Cel:} Utworzenie maszyny wirtualnej z systemem Ubuntu w stworzonej Resource Group.

\begin{enumerate}
    \item W Resource Group \texttt{example-rg}, kliknij \texttt{+ Add}, a następnie wybierz \texttt{Virtual Machine}.
    \item Wprowadź następujące dane:
    \begin{itemize}
        \item \textbf{Name:} \texttt{example-vm}.
        \item \textbf{Region:} użyj innych regionów niż Europe West, może być przepełniony.
        \item \textbf{Image:} \texttt{Ubuntu Server 24.04 LTS}.
        \item \textbf{Size:} poszukaj czegoś taniego, najlepiej \texttt{B1s}.
        \item \textbf{Authentication type:} Hasło lub klucz SSH.
    \end{itemize}
    \item Kliknij \texttt{Review + Create}, a następnie \texttt{Create}.
    \item Po utworzeniu, połącz się z maszyną wirtualną za pomocą SSH:
    \begin{lstlisting}
ssh azureuser@<public-ip>
    \end{lstlisting}
    \item Sprawdź działanie maszyny wirtualnej, np. wykonując polecenie \texttt{uptime}.
    \item Usuń wszystkie zasoby, aby uniknąć kosztów. Najlepiej usunąć całą \texttt{Resource Group}.
\end{enumerate}


\section{Wprowadzenie do Terraform}
Terraform to narzędzie typu Infrastructure as Code (IaC), które umożliwia deklaratywne zarządzanie infrastrukturą w chmurze, w tym w Microsoft Azure. W tej sekcji studenci poznają podstawy działania Terraform oraz nauczą się przygotowywać środowisko do pracy z tym narzędziem.

\subsection{Podstawowe informacje o Terraform}
\begin{itemize}
    \item \textbf{Czym jest Terraform?}
    \begin{itemize}
        \item Narzędzie open-source opracowane przez firmę HashiCorp.
        \item Pozwala na deklaratywne definiowanie infrastruktury w plikach konfiguracyjnych.
        \item Wspiera wiele dostawców chmurowych, takich jak Azure, AWS, Google Cloud.
    \end{itemize}
    \item \textbf{Zalety Terraform:}
    \begin{itemize}
        \item Powtarzalność konfiguracji – eliminacja błędów manualnych.
        \item Możliwość wersjonowania infrastruktury.
        \item Szybkie skalowanie i zarządzanie zasobami.
    \end{itemize}
\end{itemize}

\subsection{Instalacja Terraform}
\textbf{Kroki instalacji Terraform:}
\begin{enumerate}
    \item Pobierz Terraform z oficjalnej strony: \url{https://www.terraform.io/downloads}.
    \item Rozpakuj archiwum i skopiuj plik wykonywalny do katalogu dostępnego w \texttt{PATH}, np. \texttt{/usr/local/bin}.
    \item Sprawdź instalację, uruchamiając polecenie:
    \begin{lstlisting}
terraform version
    \end{lstlisting}
\end{enumerate}

\subsection{Przygotowanie środowiska do pracy z Azure}
Terraform wymaga integracji z Microsoft Azure, co można osiągnąć za pomocą narzędzia Azure CLI. Poniższe kroki można pominąć przez wybranie Azure Cloud Shell (ikonka terminala po prawej stronie wyszukiwarki w Azure Portal). Wszystkie polecenia są już zainstalowane w Cloud Shell. Pamiętaj o konfiguracji subskrypcji, jeśli masz więcej niż jedną.

\begin{enumerate}
    \item Zainstaluj Azure CLI, postępując zgodnie z instrukcją na stronie: \url{https://learn.microsoft.com/en-us/cli/azure/install-azure-cli}.
    \item Zaloguj się do Azure za pomocą polecenia:
    \begin{lstlisting}
az login
    \end{lstlisting}
    \item Skonfiguruj domyślną subskrypcję (opcjonalnie):
    \begin{lstlisting}
az account set --subscription <subscription-id>
    \end{lstlisting}
\end{enumerate}

\subsection{Pierwsze kroki z Terraform}
\textbf{Tworzenie projektu Terraform:}
\begin{enumerate}
    \item Utwórz nowy katalog projektu:
    \begin{lstlisting}
mkdir terraform-azure && cd terraform-azure
    \end{lstlisting}
    \item Utwórz plik \texttt{main.tf} z następującą konfiguracją:
    \begin{lstlisting}
provider "azurerm" {
  features {}
}

resource "azurerm_resource_group" "example" {
  name     = "example-rg"
  location = "West Europe" 
}
    \end{lstlisting}
    \item Zainicjalizuj Terraform w katalogu projektu:
    \begin{lstlisting}
terraform init
    \end{lstlisting}
    \item Sprawdź, jakie zmiany zostaną wprowadzone:
    \begin{lstlisting}
terraform plan
    \end{lstlisting}
    \item Utwórz Resource Group, wykonując polecenie:
    \begin{lstlisting}
terraform apply
    \end{lstlisting}
    \item Po zakończeniu pracy usuń zasoby:
    \begin{lstlisting}
terraform destroy
    \end{lstlisting}
\end{enumerate}


\section{Zmienne i zaawansowane przykłady w Terraform}
Terraform umożliwia dynamiczne zarządzanie konfiguracją za pomocą zmiennych. Dzięki temu można zwiększyć elastyczność i powtarzalność kodu. W tej sekcji studenci nauczą się definiować zmienne, używać ich w zasobach oraz tworzyć bardziej zaawansowane konfiguracje.

\subsection{Definiowanie zmiennych w Terraform}
\textbf{Zmienne w Terraform:}
\begin{itemize}
    \item Zmienne są definiowane w pliku \texttt{variables.tf}.
    \item Umożliwiają dynamiczne przypisywanie wartości do konfiguracji.
    \item Mogą być przekazywane jako parametry wiersza poleceń, pliki \texttt{.tfvars} lub mieć wartości domyślne.
\end{itemize}

\textbf{Przykład definicji zmiennej:}
\begin{lstlisting}
variable "resource_group_name" {
  default = "example-rg"
}

variable "location" {
  default = "East US"
}
\end{lstlisting}

\textbf{Użycie zmiennych w konfiguracji:}
\begin{lstlisting}
resource "azurerm_resource_group" "example" {
  name     = var.resource_group_name
  location = var.location
}
\end{lstlisting}

\subsection{Ćwiczenie: Tworzenie Resource Group za pomocą zmiennych}
\begin{enumerate}
    \item Utwórz plik \texttt{variables.tf} i dodaj następujące zmienne:
    \begin{lstlisting}
variable "resource_group_name" {
  default = "example-rg"
}

variable "location" {
  default = "East US"
}
    \end{lstlisting}
    \item Zaktualizuj plik \texttt{main.tf}, aby używał zmiennych:
    \begin{lstlisting}
resource "azurerm_resource_group" "example" {
  name     = var.resource_group_name
  location = var.location
}
    \end{lstlisting}
    \item Zainicjalizuj Terraform:
    \begin{lstlisting}
terraform init
    \end{lstlisting}
    \item Przeanalizuj plan:
    \begin{lstlisting}
terraform plan
    \end{lstlisting}
    \item Utwórz Resource Group:
    \begin{lstlisting}
terraform apply
    \end{lstlisting}
\end{enumerate}

\subsection{Zaawansowane przykłady: Tworzenie maszyn wirtualnych z dynamiczną konfiguracją}
\textbf{Cel:} Wykorzystanie zmiennych do tworzenia kilku maszyn wirtualnych w oparciu o dynamiczną listę.

\textbf{Przykład konfiguracji:}
Zaczynamy od stworzenia warstwy sieciowej w Microsoft Azure. Użyjemy do tego \texttt{Resource Group} z poprzednich ćwiczeń oraz \texttt{Virtual Network} i \texttt{Subnet}.
\begin{lstlisting}
resource "azurerm_virtual_network" "example" {
  name                = "example-network"
  address_space       = ["10.0.0.0/16"]
  location            = azurerm_resource_group.example.location
  resource_group_name = azurerm_resource_group.example.name
}
resource "azurerm_subnet" "example" {
  name                 = "internal"
  resource_group_name  = azurerm_resource_group.example.name
  virtual_network_name = azurerm_virtual_network.example.name
  address_prefixes     = ["10.0.2.0/24"]
}

\end{lstlisting}

Przetestuj konfigurację, wykonując polecenie \texttt{terraform apply}. Zauważ referencję zasobu \texttt{Subnet} do nazwy \texttt{Virtual Network}.

Następnie utworzymy interfejsy sieciowe dla 3 maszyn wirtualnych. Użyjemy do tego parametru \texttt{count}, który pozwala na tworzenie wielu instancji zasobu na podstawie jednej definicji. Iterator \texttt{count.index} pozwala na dynamiczne przypisanie adresów IP i rozróżnienie instancji.

\begin{lstlisting}
resource "azurerm_network_interface" "example" {
  name                = "example-nic-${count.index}"
  location            = azurerm_resource_group.example.location
  resource_group_name = azurerm_resource_group.example.name

  ip_configuration {
    name                          = "internal"
    subnet_id                     = azurerm_subnet.example.id
    private_ip_address_allocation = "Dynamic"
  }
}
\end{lstlisting}

Zaaplikuj konfigurację i obserwuj jak tworzą się interfejsy sieciowe. Następnie przy pomocy dokumentacji dostępnej na \url{https://registry.terraform.io} znajdź dokumentację dla providera \texttt{azurerm} i poszukaj jak napisać definicję maszyny wirtualnej. Przy pisaniu definicji uwzględniej parametr \texttt{count} i odnieś się do interfejsów sieciowych przez odpowiedni parametr, np. \texttt{network\_interface\_ids}.

\subsection{Wykorzystanie zmiennych w sieciach i Storage Account}
\textbf{Przykład zmiennych dla sieci:}
\begin{lstlisting}
variable "vnet_address_space" {
  default = ["10.0.0.0/16"]
}

resource "azurerm_virtual_network" "vnet" {
  name                = "example-vnet"
  address_space       = var.vnet_address_space
  location            = azurerm_resource_group.example.location
  resource_group_name = azurerm_resource_group.example.name
}
\end{lstlisting}

\textbf{Ćwiczenie:}
\begin{enumerate}
    \item Skonfiguruj zmienne dla sieci (\texttt{address\_space}) i Subnet (\texttt{address\_prefixes}).
    \item Użyj zmiennych w definicji zasobów.
    \item Przeprowadź symulację zmian (\texttt{terraform plan}) i wdrożenie (\texttt{terraform apply}).
\end{enumerate}


\section{Relacje między zasobami w Terraform – Storage Account i VM}
Terraform umożliwia tworzenie zależności między zasobami w sposób deklaratywny. Dzięki temu można łatwo integrować różne usługi w chmurze. W tej sekcji studenci nauczą się, jak połączyć Storage Account z maszyną wirtualną w ramach jednej infrastruktury.

\subsection{Przykład relacji: Storage Account i maszyna wirtualna}
\textbf{Cel:} Utworzenie Storage Account, a następnie skonfigurowanie maszyny wirtualnej, aby korzystała z klucza dostępu do tego Storage Account.

\subsection{Konfiguracja Storage Account}
\begin{lstlisting}
resource "azurerm_storage_account" "storage" {
  name                     = "examplestorage"
  resource_group_name      = azurerm_resource_group.example.name
  location                 = azurerm_resource_group.example.location
  account_tier             = "Standard"
  account_replication_type = "LRS"

  tags = {
    environment = "dev"
  }
}
\end{lstlisting}

\subsection{Konfiguracja maszyny wirtualnej z dostępem do Storage Account}
\begin{lstlisting}
resource "azurerm_virtual_machine" "example" {
  name                  = "example-vm"
  location              = azurerm_resource_group.example.location
  resource_group_name   = azurerm_resource_group.example.name
  network_interface_ids = [azurerm_network_interface.example.id]
  vm_size               = "Standard_B1s"

  os_disk {
    caching              = "ReadWrite"
    storage_account_type = "Standard_LRS"
  }

  source_image_reference {
    publisher = "Canonical"
    offer     = "UbuntuServer"
    sku       = "18.04-LTS"
    version   = "latest"
  }

  admin_username = "azureuser"
  admin_password = "P@ssw0rd1234!"

  tags = {
    environment = "dev"
  }

  provisioner "local-exec" {
    command = <<EOT
      echo "STORAGE_KEY=${azurerm_storage_account.storage.primary_access_key}" >> /etc/environment
    EOT
  }
}
\end{lstlisting}

\subsection{Ćwiczenie: Utworzenie infrastruktury i weryfikacja połączenia}
\begin{enumerate}
    \item Utwórz plik \texttt{main.tf} zawierający Resource Group, Storage Account oraz maszynę wirtualną.
    \item Użyj klucza dostępu do Storage Account w konfiguracji maszyny wirtualnej:
    \begin{itemize}
        \item Klucz \texttt{primary\_access\_key} jest odczytywany dynamicznie z zasobu \texttt{azurerm\_storage\_account}.
        \item Zmienna środowiskowa \texttt{STORAGE\_KEY} zostaje zapisana w systemie maszyny wirtualnej.
    \end{itemize}
    \item Wykonaj \texttt{terraform apply}, aby wdrożyć infrastrukturę.
    \item Połącz się z maszyną wirtualną za pomocą SSH:
    \begin{lstlisting}
ssh azureuser@<public-ip>
    \end{lstlisting}
    \item Zweryfikuj obecność zmiennej środowiskowej:
    \begin{lstlisting}
cat /etc/environment | grep STORAGE_KEY
    \end{lstlisting}
\end{enumerate}

\subsection{Przykład z użyciem dynamicznych zmiennych dla Storage Account}
Aby ułatwić skalowanie i elastyczność, można zdefiniować zmienną na poziomie nazwy Storage Account:
\begin{lstlisting}
variable "storage_account_name" {
  default = "examplestorage"
}

resource "azurerm_storage_account" "storage" {
  name                     = var.storage_account_name
  resource_group_name      = azurerm_resource_group.example.name
  location                 = azurerm_resource_group.example.location
  account_tier             = "Standard"
  account_replication_type = "LRS"
}
\end{lstlisting}

\textbf{Ćwiczenie:} Zaktualizuj konfigurację Terraform tak, aby nazwa Storage Account była przekazywana jako zmienna. Przetestuj wprowadzone zmiany, uruchamiając:
\begin{lstlisting}
terraform plan
terraform apply
\end{lstlisting}


\section{Konfiguracja PostgreSQL i integracja z maszyną wirtualną}
W tej sekcji studenci nauczą się tworzyć zarządzaną bazę danych PostgreSQL w Microsoft Azure oraz łączyć ją z maszyną wirtualną. W ramach zadania studenci skonfigurują serwer PostgreSQL, dostosują reguły zapory sieciowej oraz przetestują połączenie z poziomu maszyny wirtualnej.

\subsection{Tworzenie zarządzanego serwera PostgreSQL}
\textbf{Cel:} Utworzenie zarządzanej bazy danych PostgreSQL w Azure.

\begin{lstlisting}
resource "azurerm_postgresql_server" "postgresql" {
  name                = "example-postgresql"
  location            = azurerm_resource_group.example.location
  resource_group_name = azurerm_resource_group.example.name
  administrator_login = "psqladmin"
  administrator_login_password = "P@ssw0rd1234!"
  sku_name            = "GP_Gen5_2"
  version             = "11"

  storage_mb = 5120
  ssl_enforcement_enabled = false
  auto_grow_enabled = true
}

resource "azurerm_postgresql_database" "exampledb" {
  name                = "exampledb"
  resource_group_name = azurerm_resource_group.example.name
  server_name         = azurerm_postgresql_server.postgresql.name
  charset             = "UTF8"
  collation           = "English_United States.1252"
}
\end{lstlisting}

\subsection{Konfiguracja reguł zapory sieciowej PostgreSQL}
\textbf{Cel:} Skonfigurowanie dostępu do serwera PostgreSQL dla maszyny wirtualnej.

\begin{lstlisting}
resource "azurerm_postgresql_firewall_rule" "allow_vm" {
  name                = "allow-vm"
  resource_group_name = azurerm_resource_group.example.name
  server_name         = azurerm_postgresql_server.postgresql.name
  start_ip_address    = azurerm_public_ip.vm.public_ip_address
  end_ip_address      = azurerm_public_ip.vm.public_ip_address
}
\end{lstlisting}

\textbf{Wyjaśnienie:}
\begin{itemize}
    \item Reguła zapory zezwala na połączenie do PostgreSQL z określonego adresu IP.
    \item W tym przypadku IP publiczne maszyny wirtualnej (\texttt{azurerm\_public\_ip.vm.public\_ip\_address}) jest używane jako źródło.
\end{itemize}

\subsection{Integracja maszyny wirtualnej z PostgreSQL}
\textbf{Cel:} Skonfigurowanie maszyny wirtualnej do połączenia z PostgreSQL.

\begin{lstlisting}
resource "azurerm_virtual_machine" "example" {
  name                  = "example-vm"
  location              = azurerm_resource_group.example.location
  resource_group_name   = azurerm_resource_group.example.name
  network_interface_ids = [azurerm_network_interface.example.id]
  vm_size               = "Standard_B1s"

  os_disk {
    caching              = "ReadWrite"
    storage_account_type = "Standard_LRS"
  }

  source_image_reference {
    publisher = "Canonical"
    offer     = "UbuntuServer"
    sku       = "18.04-LTS"
    version   = "latest"
  }

  admin_username = "azureuser"
  admin_password = "P@ssw0rd1234!"

  provisioner "remote-exec" {
    inline = [
      "sudo apt update",
      "sudo apt install postgresql-client -y",
      "echo 'export DB_HOST=${azurerm_postgresql_server.postgresql.fqdn}' >> ~/.bashrc",
      "echo 'export DB_USER=psqladmin' >> ~/.bashrc",
      "echo 'export DB_PASSWORD=P@ssw0rd1234!' >> ~/.bashrc",
      "source ~/.bashrc"
    ]
  }
}
\end{lstlisting}

\textbf{Wyjaśnienie:}
\begin{itemize}
    \item Klient PostgreSQL jest instalowany na maszynie wirtualnej.
    \item Zmienna środowiskowa \texttt{DB\_HOST} zawiera pełną nazwę domenową PostgreSQL (\texttt{fqdn}).
    \item Użytkownik i hasło są zapisane jako zmienne środowiskowe.
\end{itemize}

\subsection{Ćwiczenie: Testowanie połączenia z PostgreSQL}
\begin{enumerate}
    \item Po wdrożeniu infrastruktury połącz się z maszyną wirtualną za pomocą SSH:
    \begin{lstlisting}
ssh azureuser@<public-ip>
    \end{lstlisting}
    \item Zweryfikuj zmienne środowiskowe:
    \begin{lstlisting}
echo $DB_HOST
echo $DB_USER
echo $DB_PASSWORD
    \end{lstlisting}
    \item Przetestuj połączenie z bazą danych za pomocą klienta PostgreSQL:
    \begin{lstlisting}
psql -h $DB_HOST -U $DB_USER -d exampledb
    \end{lstlisting}
    \item Wylistuj dostępne tabele w bazie danych:
    \begin{lstlisting}
\dt
    \end{lstlisting}
\end{enumerate}

\subsection{Podsumowanie zadania}
\textbf{Cele osiągnięte w tym zadaniu:}
\begin{itemize}
    \item Tworzenie zarządzanej bazy danych PostgreSQL.
    \item Konfiguracja reguł zapory sieciowej.
    \item Integracja maszyny wirtualnej z bazą danych PostgreSQL.
    \item Testowanie połączenia z bazy danych za pomocą klienta \texttt{psql}.
\end{itemize}


\section{Podsumowanie}
Podczas tych zajęć zostały omówione kluczowe zagadnienia związane z wykorzystaniem Terraform oraz integracją zasobów w chmurze Microsoft Azure. Główne punkty realizowane podczas zajęć to:

\begin{itemize}
    \item \textbf{Podstawy platformy Microsoft Azure:} Tworzenie Resource Group oraz zarządzanie zasobami za pomocą Azure Portal i Azure CLI.
    \item \textbf{Podstawy Terraform:} Wprowadzenie do Infrastructure as Code, instalacja Terraform i jego integracja z Azure.
    \item \textbf{Tworzenie zasobów:} Automatyzacja tworzenia Resource Group, maszyn wirtualnych, sieci i Storage Account przy użyciu Terraform.
    \item \textbf{Relacje między zasobami:} Łączenie zasobów, takich jak Storage Account i maszyny wirtualne, za pomocą referencji.
    \item \textbf{Zarządzana baza danych PostgreSQL:} Konfiguracja serwera PostgreSQL w Azure, reguł zapory sieciowej oraz integracja z maszyną wirtualną.
\end{itemize}

\subsection{Wnioski}
Terraform pozwala na deklaratywne zarządzanie infrastrukturą, co znacząco upraszcza procesy wdrożeniowe w systemach rozproszonych. Kluczowe wnioski:
\begin{itemize}
    \item Automatyzacja tworzenia zasobów w chmurze pozwala na większą efektywność i eliminację błędów manualnych.
    \item Integracja różnych typów zasobów, takich jak bazy danych i maszyny wirtualne, umożliwia budowę skalowalnej infrastruktury.
    \item Testowanie połączeń i funkcjonalności zintegrowanych zasobów pozwala na weryfikację poprawności konfiguracji.
\end{itemize}

\subsection{Kroki na przyszłość}
Aby poszerzyć wiedzę, warto:
\begin{itemize}
    \item Eksperymentować z bardziej zaawansowanymi funkcjami Terraform, takimi jak moduły czy zdalny stan.
    \item Poznać inne usługi Azure, takie jak Load Balancer czy Azure Kubernetes Service (AKS).
    \item Zapoznać się z technikami monitorowania zasobów w chmurze oraz integracji z narzędziami, takimi jak Prometheus.
\end{itemize}

Podczas zajęć zdobyto praktyczne umiejętności w zarządzaniu infrastrukturą chmurową, co jest kluczowym elementem nowoczesnych metod zarządzania systemami rozproszonymi.
\end{document}
