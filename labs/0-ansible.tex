\documentclass{article}
\usepackage[utf8]{inputenc}
\usepackage{hyperref}
\usepackage{listings}
\usepackage{color}
\setlength{\parindent}{0pt}

\title{Zarządzanie Systemami Rozproszonymi\\Laboratoria z Ansible}
\author{mgr inż. Jakub Woźniak}
\date{}

\begin{document}

\maketitle

\section*{Wprowadzenie}

\textbf{Cel laboratoriów}:  
Celem tych laboratoriów jest zapoznanie studentów z narzędziem Ansible oraz automatyzacją zarządzania konfiguracją. Studenci będą używać Ansible do instalacji oprogramowania, zarządzania plikami oraz konfiguracji systemów w sieci za pomocą playbooków YAML. Laboratoria kończą się zadaniem zbudowania klastra Redis z jednym węzłem Master i dwoma węzłami Slave.

\textbf{Wymagana wiedza}:  
\begin{itemize}
    \item Znajomość podstaw systemu Linux (komendy terminala).
    \item Podstawowa znajomość sieci (np. SSH, IP).
\end{itemize}

\textbf{Narzędzia}:  
\begin{itemize}
    \item Maszyna kontrolna (np. openSUSE) z zainstalowanym Ansible.
    \item Trzy maszyny wirtualne w tej samej sieci, na których będą wykonywane zadania.
    \item Połączenie SSH między maszynami.
\end{itemize}

\section{Przygotowanie środowiska}

\textbf{Teoria:}  
W środowiskach rozproszonych ważne jest zapewnienie bezproblemowej komunikacji między maszynami. Dzięki Ansible możemy automatycznie zarządzać wieloma systemami jednocześnie. W tym laboratorium studenci skonfigurują trzy maszyny wirtualne w tej samej sieci, które będą mogły komunikować się ze sobą oraz z maszyną kontrolną przez SSH.

\subsection*{Tutorial: Konfiguracja maszyn wirtualnych w VirtualBox z użyciem gotowych obrazów}

1. \textbf{Pobranie gotowych obrazów systemu:}
   \begin{itemize}
       \item \textbf{Ubuntu Minimal}: Pobierz obraz minimalny Ubuntu z \href{https://cloud-images.ubuntu.com/}{Ubuntu Cloud}.
       \item \textbf{openSUSE Minimal}: Pobierz minimalny obraz openSUSE z \href{https://get.opensuse.org/tumbleweed/}{openSUSE Tumbleweed}.
   \end{itemize}

2. \textbf{Konfiguracja sieci:}
   \begin{itemize}
       \item Utwórz sieć typu "Host-Only" w VirtualBox, która pozwala na komunikację między maszynami.
       \item Dla każdej maszyny wirtualnej skonfiguruj Adapter 1 jako NAT (dostęp do internetu) i Adapter 2 jako Host-Only.
   \end{itemize}

3. \textbf{Importowanie maszyn wirtualnych:}
   \begin{itemize}
       \item Utwórz nowe maszyny w VirtualBox, korzystając z pobranych obrazów jako dysków.
   \end{itemize}

4. \textbf{Konfiguracja bezhasłowego SSH:}
   \begin{itemize}
       \item Na maszynie hosta wygeneruj parę kluczy SSH za pomocą \texttt{ssh-keygen -t rsa -b 4096}.
       \item Skopiuj klucz publiczny na każdą z maszyn wirtualnych, korzystając z \texttt{ssh-copy-id}.
   \end{itemize}

5. \textbf{Sprawdzenie komunikacji między maszynami:}
   \begin{itemize}
       \item Użyj polecenia \texttt{ping}, aby upewnić się, że maszyny wirtualne mogą się ze sobą komunikować.
   \end{itemize}

\section{Sprawdzenie połączenia}

\textbf{Teoria:}  
Pierwszym krokiem przy pracy z Ansible jest upewnienie się, że maszyna kontrolna może komunikować się z innymi maszynami. Ansible umożliwia to za pomocą modułu \texttt{ping}, który testuje dostępność maszyn w grupie.

\textbf{Polecenie \texttt{ping}:}

\noindent\begin{lstlisting}
ansible myservers -m ping
\end{lstlisting}

To polecenie wysyła żądanie ping do każdej maszyny w grupie \texttt{myservers}, sprawdzając, czy jest dostępna.

\section{Zbieranie faktów}

\textbf{Teoria:}  
Moduł \texttt{gather\_facts} pozwala zebrać szczegółowe informacje o maszynach, takie jak adresy IP, system operacyjny czy dostępne zasoby. Jest to przydatne przy dalszej automatyzacji.

\textbf{Zbieranie faktów z maszyn:}

\noindent\begin{lstlisting}
ansible myservers -m gather_facts
\end{lstlisting}

Wyświetli to szczegółowe dane na temat każdej z maszyn w formacie JSON.

\section{Instalacja Nginx za pomocą playbooka}

\textbf{Teoria:}  
Playbooki Ansible to pliki YAML, które definiują zestawy instrukcji do wykonania na zdalnych maszynach. W tym kroku zainstalujemy serwer Nginx na wszystkich maszynach.

\textbf{Playbook YAML do instalacji Nginx:}

\noindent\begin{lstlisting}
---
- hosts: myservers
  become: yes
  tasks:
    - name: Install Nginx
      apt:
        name: nginx
        state: present
\end{lstlisting}

\textbf{Uruchomienie playbooka:}

\noindent\begin{lstlisting}
ansible-playbook nginx-install.yml
\end{lstlisting}

\section{Modyfikacja strony głównej za pomocą playbooka}

\textbf{Teoria:}  
Ansible umożliwia automatyczne zarządzanie plikami na zdalnych maszynach. W tym kroku skopiujemy plik \texttt{index.html} na wszystkie maszyny, zmieniając stronę główną Nginx.

\textbf{Playbook YAML do skopiowania pliku \texttt{index.html}:}

\noindent\begin{lstlisting}
---
- hosts: myservers
  become: yes
  tasks:
    - name: Copy index.html to all servers
      copy:
        src: /path/to/index.html
        dest: /var/www/html/index.html
\end{lstlisting}

\textbf{Uruchomienie playbooka:}

\noindent\begin{lstlisting}
ansible-playbook update-index.yml
\end{lstlisting}

\section{Wykorzystanie szablonu do dynamicznej strony}

\textbf{Teoria:}  
Ansible pozwala na korzystanie z szablonów, które dynamicznie generują pliki na podstawie zmiennych. W tym kroku stworzymy szablon \texttt{index.html}, który będzie wyświetlał adres IP każdej maszyny.

\textbf{Szablon \texttt{index.html.j2} z dynamicznymi zmiennymi:}

\noindent\begin{lstlisting}
<html>
  <head><title>Ansible Lab</title></head>
  <body>
    <h1>This is {{ ansible_default_ipv4.address }}</h1>
  </body>
</html>
\end{lstlisting}

\textbf{Playbook YAML do wdrożenia szablonu:}

\noindent\begin{lstlisting}
---
- hosts: myservers
  become: yes
  tasks:
    - name: Deploy template index.html
      template:
        src: /path/to/index.html.j2
        dest: /var/www/html/index.html
\end{lstlisting}

\textbf{Uruchomienie playbooka:}

\noindent\begin{lstlisting}
ansible-playbook template-index.yml
\end{lstlisting}

\section{Instalacja Redis (Master-Slave)}

\textbf{Teoria:}  
Redis jest bazą danych typu key-value, która obsługuje replikację danych w trybie Master-Slave. W tym kroku skonfigurujemy klaster Redis z jednym węzłem Master i dwoma węzłami Slave.

\textbf{Zadanie:}  
Stwórz playbook, który zainstaluje Redis na wszystkich maszynach i skonfiguruje replikację Master-Slave. Po zakończeniu zadania sprawdź, czy dane są replikowane między węzłami.

\textbf{Kryteria zaliczenia:}
\begin{itemize}
    \item Poprawnie skonfigurowany klaster Redis z jednym Master i dwoma Slave.
    \item Weryfikacja, że dane dodane na węźle Master są replikowane na węzłach Slave.
\end{itemize}

\end{document}

