\documentclass{article}
\usepackage[utf8]{inputenc}
\usepackage[T1]{fontenc}
\usepackage{listings}
\usepackage{graphicx}
\usepackage{fontspec}
\usepackage{xurl}
\usepackage[pdftex]{hyperref}
\setlength{\parindent}{0pt}
\usepackage{xcolor}


\lstdefinelanguage{HCL}{
  morekeywords={resource,variable,output,provider,module,terraform,backend},
  sensitive=true,
  morestring=[b]"
}

\lstdefinestyle{TerraformStyle}{
  basicstyle=\ttfamily\footnotesize,
  keywordstyle=\color{blue}\bfseries,
  commentstyle=\color{teal},
  stringstyle=\color{orange},
  numbers=left,
  numbersep=5pt,
  numberstyle=\tiny\color{gray},
  frame=single,
  breaklines=true,
  columns=fullflexible,
  keepspaces=true,
  language=HCL
}

\lstset{style=TerraformStyle}

\title{Zarządzanie Systemami Rozproszonymi\\Laboratoria z Terraform}
\author{mgr inż. Jakub Woźniak}

\begin{document}
\maketitle

\section{Wprowadzenie do zaawansowanych funkcji Terraform}
Terraform to narzędzie służące do zarządzania infrastrukturą w chmurze z wykorzystaniem podejścia Infrastructure as Code (IaC). Zaawansowane funkcje Terraform, takie jak wykorzystanie modułów, zmiennych oraz zdalnego stanu, pozwalają na budowę skalowalnych i czytelnych środowisk w chmurze. 

\subsection{Moduły}
\textbf{Moduł} w Terraform to zestaw plików, w których można zdefiniować powiązane ze sobą zasoby i ich konfiguracje. Dzięki modułom:
\begin{itemize}
    \item Ułatwia się powtórne wykorzystanie kodu.
    \item Redukuje się złożoność dzięki podziałowi infrastruktury na mniejsze części.
    \item Łatwiej utrzymać spójność oraz standardy w różnych częściach projektu.
\end{itemize}

\subsection{Zmiennie}
\textbf{Zmienna} (\texttt{variable}) jest wykorzystywana w Terraform do parametryzacji modułów i skryptów. Można określać typy zmiennych oraz domyślne wartości. Przekazywanie wartości do zmiennych odbywa się najczęściej przez pliki \texttt{.tfvars} lub parametry CLI.

\subsection{Zdalny stan (Remote State)}
\textbf{Stan Terraform} to plik (\texttt{.tfstate}), w którym przechowywana jest informacja o wdrożonych zasobach. Domyślnie stan zapisywany jest w pliku lokalnym, ale w przypadku zespołowej pracy nad infrastrukturą niezbędne jest przeniesienie stanu do lokalizacji zdalnej (np. Azure Blob Storage), aby uniknąć konfliktów i umożliwić współdzieloną edycję.

\subsection{Dlaczego warto stosować modularność w Terraform}
\begin{itemize}
    \item \textbf{Reużywalność kodu}: Te same wzorce infrastruktury można wykorzystać w wielu projektach.
    \item \textbf{Łatwość utrzymania}: Modyfikacje w module są automatycznie dostępne we wszystkich miejscach, w których moduł jest używany.
    \item \textbf{Skalowalność}: Infrastruktura może rozrastać się w uporządkowany sposób.
\end{itemize}

\vspace{5mm}

W kolejnych ćwiczeniach zostaną przedstawione przykłady praktycznego wykorzystania modułów do tworzenia infrastruktury chmurowej na platformie Azure. Każde ćwiczenie rozpoczyna się wstępem teoretycznym, który omawia kluczowe pojęcia.

\section{Ćwiczenie 1: Moduł Resource Group}

\subsection{Wprowadzenie teoretyczne}
\textbf{Resource Group} w Azure to podstawowa jednostka logiczna, w której przechowywane są zasoby. Poprzez stworzenie prostego modułu do zarządzania grupą zasobów (Resource Group) możemy zademonstrować strukturę modułu w Terraform:
\begin{itemize}
    \item \texttt{main.tf} – zawiera główną definicję modułu, w tym zasoby.
    \item \texttt{variables.tf} – przechowuje definicje zmiennych.
    \item \texttt{outputs.tf} – definiuje wartości wyjściowe (outputs) udostępniane przez moduł.
\end{itemize}

\subsection{Kroki do wykonania zadania}
\begin{enumerate}
    \item Utworzyć nowy folder o nazwie \texttt{rg\_module}.
    \item W tym folderze przygotować plik \texttt{main.tf}, który będzie definiował zasób Resource Group.
    \item Dodać plik \texttt{variables.tf} z definicjami zmiennych takich jak \texttt{rg\_name} i \texttt{rg\_location}.
    \item Dodać plik \texttt{outputs.tf} z wyjściem udostępniającym np. nazwę Resource Group.
    \item Wywołać moduł \texttt{rg\_module} z poziomu głównego pliku Terraform (\texttt{main.tf} w katalogu głównym projektu), przekazując odpowiednie wartości do zmiennych modułu.
\end{enumerate}

\subsection{Oczekiwany rezultat}
Po zakończeniu ćwiczenia powinna zostać utworzona nowa Resource Group w wybranym regionie Azure. Projekt Terraform będzie też zawierał pierwszy działający moduł.

\subsection{Przykładowy kod}
\begin{lstlisting}[language=HCL, caption={Tworzenie Resource Group w Terraform}]
resource "azurerm_resource_group" "example" {
  name     = var.rg_name
  location = var.rg_location
}
\end{lstlisting}

\section{Ćwiczenie 2: Moduł Virtual Network}

\subsection{Wprowadzenie teoretyczne}
\textbf{Virtual Network} (VNet) w Azure umożliwia tworzenie prywatnej przestrzeni adresowej dla zasobów w chmurze. W module umieścimy definicję VNet oraz Subnetów, pozwalając na zdefiniowanie:
\begin{itemize}
    \item Przestrzeni adresowej VNet (np. \texttt{10.0.0.0/16}).
    \item Liczby Subnetów oraz ich wielkości.
\end{itemize}

\subsection{Kroki do wykonania zadania}
\begin{enumerate}
    \item Utworzyć folder \texttt{vnet\_module}.
    \item W pliku \texttt{main.tf} zdefiniować zasób \texttt{azurerm\_virtual\_network} oraz \texttt{azurerm\_subnet}.
    \item W pliku \texttt{variables.tf} zdefiniować zmienne dla nazwy VNet, przestrzeni adresowej oraz parametrów Subnetów.
    \item W \texttt{outputs.tf} udostępnić ID VNet oraz Subnetów.
    \item Wywołać moduł \texttt{vnet\_module} z głównego \texttt{main.tf}, przekazując odpowiednie wartości zmiennych.
\end{enumerate}

\subsection{Oczekiwany rezultat}
Po zakończeniu ćwiczenia powinna zostać utworzona sieć wirtualna z Subnetami według zadanych parametrów. Nowy moduł umożliwi łatwą konfigurację warstwy sieciowej dla kolejnych zasobów.

\section{Ćwiczenie 3: Moduł Virtual Machine}

\subsection{Wprowadzenie teoretyczne}
\textbf{Virtual Machine} (VM) w Azure można skonfigurować za pomocą Terraform, definiując m.in.:
\begin{itemize}
    \item Rozmiar maszyny (np. \texttt{Standard\_B1s}).
    \item System operacyjny (np. Linux Ubuntu).
    \item Sieć, do której VM jest podłączona.
\end{itemize}
W tym ćwiczeniu utworzymy moduł zawierający definicję maszyny wirtualnej z dynamiczną konfiguracją (np. listą maszyn, które mają być zbudowane w pętli \texttt{for\_each} lub \texttt{count}).

\subsection{Kroki do wykonania zadania}
\begin{enumerate}
    \item Utworzyć folder \texttt{vm\_module}.
    \item W pliku \texttt{main.tf} zdefiniować zasób \texttt{azurerm\_virtual\_machine} z wykorzystaniem \texttt{for\_each} lub \texttt{count}.
    \item Zdefiniować odpowiednie zmienne w \texttt{variables.tf} (np. lista maszyn, parametry systemu operacyjnego).
    \item Zdefiniować wartości wyjściowe w \texttt{outputs.tf} (np. adresy IP utworzonych VM).
    \item Wywołać moduł \texttt{vm\_module} z głównego \texttt{main.tf}, tworząc jedną lub wiele maszyn wirtualnych.
\end{enumerate}

\subsection{Oczekiwany rezultat}
Po zakończeniu ćwiczenia w wybranej sieci wirtualnej zostaną utworzone maszyny wirtualne zdefiniowane w module, a wszystkie parametry (np. rodzaj systemu czy liczba maszyn) będą kontrolowane za pomocą zmiennych.

\section{Ćwiczenie 4: Połączenie modułów – Kompleksowa infrastruktura}

\subsection{Wprowadzenie teoretyczne}
W tym ćwiczeniu zintegrujemy wszystkie dotychczasowe moduły w jednym projekcie. Celem jest stworzenie kompletnej infrastruktury obejmującej:
\begin{itemize}
    \item Resource Group.
    \item Virtual Network z Subnetami.
    \item Maszyny wirtualne.
    \item Load Balancer (opcjonalnie), aby rozkładać ruch do maszyn.
\end{itemize}

\subsection{Kroki do wykonania zadania}
\begin{enumerate}
    \item W głównym pliku \texttt{main.tf} (poza modułami) zdefiniować nową Resource Group lub wykorzystać istniejącą z poprzednich ćwiczeń.
    \item Wywołać moduł VNet z podaniem przestrzeni adresowej i Subnetów.
    \item Wywołać moduł VM, aby utworzyć klaster maszyn wirtualnych w Subnetach.
    \item Opcjonalnie dodać moduł Load Balancer, przekazując mu informacje o maszynach.
\end{enumerate}

\subsection{Oczekiwany rezultat}
W efekcie powstaje zestaw zasobów, które mogą wspólnie stanowić środowisko testowe lub podstawę produkcyjnej infrastruktury.

\section{Ćwiczenie 5: Zarządzana baza danych PostgreSQL}

\subsection{Wprowadzenie teoretyczne}
Zarządzane bazy danych w Azure (np. \texttt{azurerm\_postgresql\_server} i \texttt{azurerm\_postgresql\_database}) pozwalają na łatwą konfigurację, automatyczne aktualizacje i wysoki poziom dostępności. Terraform umożliwia definiowanie parametrów instancji bazy i kontrolowanie dostępu sieciowego.

\subsection{Kroki do wykonania zadania}
\begin{enumerate}
    \item Utworzyć folder \texttt{postgresql\_module}.
    \item W pliku \texttt{main.tf} zdefiniować:
    \begin{itemize}
        \item \texttt{azurerm\_postgresql\_server} – definiujący serwer bazy danych.
        \item \texttt{azurerm\_postgresql\_database} – tworzący konkretną bazę danych.
        \item Ewentualne reguły firewall lub sieciowe dla bazy.
    \end{itemize}
    \item Zdefiniować zmienne w \texttt{variables.tf} (hasło, SKU bazy, nazwa bazy, itp.).
    \item Zdefiniować wartości wyjściowe w \texttt{outputs.tf} (np. connection string).
    \item Powiązać moduł bazy danych z maszynami wirtualnymi, aby aplikacja mogła łączyć się z bazą.
\end{enumerate}

\subsection{Oczekiwany rezultat}
Po zakończeniu ćwiczenia powinna zostać utworzona zarządzana baza danych PostgreSQL, a maszyny wirtualne będą miały możliwość połączenia z nią. 

\section{Ćwiczenie 6: Zdalny stan Terraform}

\subsection{Wprowadzenie teoretyczne}
Konfiguracja \textbf{zdalnego stanu} umożliwia bezpieczne przechowywanie pliku stanu w chmurze. Dzięki temu wiele osób może pracować jednocześnie z tym samym stanem, a Terraform zapewnia mechanizmy blokowania zapisu i rozwiązywania konfliktów.

\subsection{Kroki do wykonania zadania}
\begin{enumerate}
    \item W pliku \texttt{backend.tf} (lub w \texttt{terraform\{...\}} sekcji głównego \texttt{main.tf}) skonfigurować \texttt{backend} typu \texttt{azurerm}.
    \item Utworzyć kontener w Azure Blob Storage, w którym będzie przechowywany stan (\texttt{tfstate}).
    \item Skonfigurować \texttt{storage\_account\_name}, \texttt{container\_name} oraz \texttt{key} (nazwa pliku stanu).
    \item Przetestować współdzielenie stanu, uruchamiając polecenia \texttt{terraform plan} i \texttt{terraform apply} z różnych stanowisk.
\end{enumerate}

\subsection{Oczekiwany rezultat}
Po zakończeniu ćwiczenia plik stanu będzie przechowywany w usłudze Azure Blob Storage, co umożliwi zespołowe zarządzanie zasobami i wyeliminuje konflikty związane z lokalnymi kopiami stanu.

\section{Ćwiczenie końcowe: Własny projekt zespołowy}

\subsection{Wprowadzenie teoretyczne}
W ćwiczeniu końcowym uczestnicy łączą wszystkie elementy poznane wcześniej, tworząc kompleksowy projekt infrastruktury w oparciu o moduły i zdalny stan. 

\subsection{Kroki do wykonania zadania}
\begin{enumerate}
    \item Zaplanować architekturę aplikacji w zespole (Resource Group, VNet, VM, Baza PostgreSQL, Load Balancer lub inne).
    \item Utworzyć folder projektu z głównym plikiem \texttt{main.tf} i plikami \texttt{tfvars}, które zawierają wartości dla zmiennych.
    \item Wykorzystać utworzone wcześniej moduły (Resource Group, VNet, VM, PostgreSQL).
    \item Skonfigurować \texttt{backend} Terraform do zdalnego stanu, jeśli to możliwe.
    \item Zbudować i przetestować całą infrastrukturę, weryfikując poprawne działanie aplikacji.
\end{enumerate}

\subsection{Oczekiwany rezultat}
Rezultatem będzie w pełni funkcjonalna infrastruktura zbudowana we współpracy zespołowej, używająca wielokrotnie zaprojektowanych modułów i udostępniająca kompletne środowisko do uruchomienia aplikacji.

\section{Podsumowanie i wnioski}
W trakcie ćwiczeń zaprezentowano, w jaki sposób Terraform umożliwia organizowanie rozbudowanej infrastruktury w sposób modularny, skalowalny i przejrzysty. Kluczowymi elementami były moduły, zmienne oraz zdalny stan, które wspólnie pomagają w utrzymaniu dużych projektów i zespołowej pracy nad kodem.

Rozszerzeniem przedstawionego podejścia może być dodanie kolejnych usług, takich jak:
\begin{itemize}
    \item \textbf{Load Balancer} (lub Application Gateway) z bardziej zaawansowaną konfiguracją reguł ruchu.
    \item \textbf{AKS} (Azure Kubernetes Service), aby konteneryzować aplikacje i zarządzać ich cyklem życia.
\end{itemize}

Niniejszy skrypt stanowi wprowadzenie do zarządzania systemami rozproszonymi z wykorzystaniem Terraform w chmurze Azure i może służyć za punkt wyjścia do dalszych eksperymentów i rozbudowy infrastruktury.

\end{document}
