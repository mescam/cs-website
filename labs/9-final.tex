\documentclass{article}
\usepackage[utf8]{inputenc}
\usepackage[T1]{fontenc}
\usepackage{listings}
\usepackage{graphicx}
\usepackage{fontspec}
\usepackage{xurl}
\usepackage[pdftex]{hyperref}
\setlength{\parindent}{0pt}
\usepackage{xcolor}


\title{Zarządzanie Systemami Rozproszonymi\\Laboratoria z Terraform i AKS}
\author{mgr inż. Jakub Woźniak}
\date{}

\lstdefinelanguage{HCL}{
  morekeywords={resource,variable,output,provider,module,terraform,backend},
  sensitive=true,
  morestring=[b]"
}

\lstset{
    basicstyle=\ttfamily\footnotesize,
    numbers=left,
    numberstyle=\tiny,
    stepnumber=1,
    numbersep=5pt,
    backgroundcolor=\color{gray!10},
    showspaces=false,
    showstringspaces=false,
    showtabs=false,
    frame=single,
    rulecolor=\color{black},
    tabsize=2,
    captionpos=b,
    breaklines=true,
    breakatwhitespace=false,
    keywordstyle=\color{blue},
    commentstyle=\color{green},
    stringstyle=\color{red},
    escapeinside={\%*}{*)},
    morekeywords={*,resource,provider,variable,module,backend}
}

\begin{document}

\maketitle

\section{Wprowadzenie}
Celem tych zajęć jest wdrożenie praktyk Infrastructure as Code (IaC) z wykorzystaniem Terraform oraz automatyzacja wdrożeń aplikacji Kubernetes przy użyciu GitOps z Argo CD. Podczas zajęć zostaną omówione i zrealizowane następujące elementy:
\begin{itemize}
    \item Tworzenie infrastruktury w chmurze Azure z Terraform.
    \item Zarządzanie klastrem AKS i aplikacjami Kubernetes.
    \item Integracja Terraform z Kubernetes przy użyciu providera Kubernetes.
    \item Automatyzacja wdrożeń za pomocą Argo CD.
\end{itemize}

\subsection{Zakres zajęć}
\begin{enumerate}
    \item Konfiguracja Terraform z remote state w Azure.
    \item Tworzenie klastra AKS.
    \item Instalacja Argo CD na klastrze Kubernetes.
    \item Tworzenie infrastruktury z Terraform z wykorzystaniem bazy danych Azure SQL.
    \item Wdrożenie aplikacji Redmine z Load Balancerem.
    \item Automatyzacja wdrożeń z Argo CD.
\end{enumerate}

\subsection{Wymagane narzędzia}
\begin{itemize}
    \item \textbf{Terraform:} Zarządzanie infrastrukturą jako kodem.
    \item \textbf{Azure CLI:} Do logowania i interakcji z Azure.
    \item \textbf{kubectl:} Zarządzanie klastrem Kubernetes.
    \item \textbf{Argo CD CLI:} Zarządzanie projektami GitOps.
    \item \textbf{Git:} Przechowywanie definicji infrastruktury.
\end{itemize}

\section{Przygotowanie środowiska}
\subsection{Konfiguracja zdalnego stanu w Terraform}
Terraform wymaga zdalnego backendu do przechowywania stanu. W Azure można użyć Azure Blob Storage.

\textbf{Kroki:}
\begin{enumerate}
    \item Utwórz Resource Group
    \item Utwórz Storage Account
    \item Skonfiguruj backend Terraform
    \item Inicjalizuj Terraform
\end{enumerate}

\section{Zadania samodzielne}
Poniżej opisano zadania do samodzielnego wykonania. 

\subsection{Tworzenie klastra AKS}
\begin{itemize}
    \item Wykorzystaj Terraform do stworzenia klastra AKS.
    \item Dokumentacja: \url{https://registry.terraform.io/providers/hashicorp/azurerm/latest/docs/resources/kubernetes_cluster}
\end{itemize}

\subsection{Instalacja Argo CD}
\begin{itemize}
    \item Zainstaluj Argo CD na klastrze AKS.
    \item Skonfiguruj dostęp do interfejsu UI Argo CD przy użyciu \texttt{kubectl port-forward}.
    \item Dokumentacja: \url{https://argo-cd.readthedocs.io/}
\end{itemize}

\subsection{Wdrożenie SQL}
\begin{itemize}
    \item Utwórz bazę danych Azure MySQL przy pomocy Terraform.
    \item Skonfiguruj reguły zapory umożliwiające dostęp z klastra Kubernetes.
\end{itemize}

\subsection{Definicje Kubernetes w Terraform}
\begin{itemize}
    \item Wykorzystaj provider Kubernetes w Terraform do utworzenia Deployment i Service dla Redmine.
    \item Skonfiguruj Load Balancer do wystawienia aplikacji na świat.
    \item Dokumentacja: \url{https://registry.terraform.io/providers/hashicorp/kubernetes/latest}
\end{itemize}

\subsection{Automatyzacja wdrożeń z Argo CD}
\begin{itemize}
    \item Skonfiguruj Argo CD, aby automatycznie wdrażał aplikację z repozytorium Git.
    \item Dokumentacja: \url{https://argo-cd.readthedocs.io/en/stable/user-guide/}
\end{itemize}

\section{Podsumowanie}
Podczas tych zajęć zrealizowano kompleksowe wdrożenie infrastruktury chmurowej w Azure, zarządzanej przy pomocy Terraform i zautomatyzowanej dzięki Argo CD. Kluczowe elementy to:
\begin{itemize}
    \item Konfiguracja i zarządzanie zdalnym stanem w Terraform.
    \item Modularne podejście do definicji infrastruktury.
    \item Automatyzacja wdrożeń aplikacji Kubernetes za pomocą GitOps.
\end{itemize}
\end{document}
