\documentclass{article}
\usepackage[utf8]{inputenc}

\usepackage{listings}
\usepackage{color}
\usepackage{graphicx}
\usepackage{hyperref}
\usepackage{geometry}
\usepackage[T1]{fontenc}
\geometry{a4paper, margin=2.5cm}

\definecolor{codegreen}{rgb}{0,0.6,0}
\definecolor{codegray}{rgb}{0.5,0.5,0.5}
\definecolor{codepurple}{rgb}{0.58,0,0.82}
\definecolor{backcolour}{rgb}{0.95,0.95,0.92}

\lstdefinestyle{mystyle}{
    backgroundcolor=\color{backcolour},   
    commentstyle=\color{codegreen},
    keywordstyle=\color{codepurple},
    stringstyle=\color{codegreen},
    basicstyle=\ttfamily\small,
    breakatwhitespace=false,         
    breaklines=true,                 
    captionpos=b,                    
    keepspaces=true,                 
    numbersep=5pt,                  
    showspaces=false,                
    showstringspaces=false,
    showtabs=false,                  
    tabsize=2
}

\lstset{style=mystyle}

\title{Laboratorium: Aplikacja Azure Functions z Terraform\\
\large Implementacja aplikacji z trzema endpointami w Pythonie}
\author{mgr inż. Jakub Woźniak}
\date{}

\begin{document}

\maketitle

\section{Wprowadzenie}

W tym laboratorium studenci stworzą aplikację bezserwerową przy użyciu Azure Functions z trzema endpointami, zaimplementowaną w języku Python. Cała infrastruktura zostanie zdefiniowana i wdrożona przy użyciu narzędzia Terraform, zgodnie z podejściem Infrastructure as Code (IaC).

\subsection{Czym są Azure Functions?}

Azure Functions to usługa przetwarzania bezserwerowego (serverless), która umożliwia uruchamianie kodu w odpowiedzi na różne zdarzenia, bez konieczności zarządzania infrastrukturą bazową. Główne zalety tej usługi:

\begin{itemize}
    \item Opłaty tylko za rzeczywiście wykorzystane zasoby
    \item Automatyczne skalowanie w zależności od obciążenia
    \item Obsługa wielu języków programowania, w tym Pythona
    \item Integracja z różnorodnymi usługami Azure i zewnętrznymi
\end{itemize}

\subsection{Czym jest Terraform?}

Terraform to narzędzie typu open-source służące do definiowania infrastruktury jako kod (IaC). Dzięki niemu można:

\begin{itemize}
    \item Deklaratywnie definiować zasoby infrastruktury
    \item Wersjonować zmiany w infrastrukturze
    \item Automatyzować wdrażanie i zarządzanie zasobami
    \item Zapewnić powtarzalność i spójność środowisk
\end{itemize}

\section{Przygotowanie środowiska laboratoryjnego}

\subsection{Wymagania wstępne}

Przed rozpoczęciem laboratorium, należy upewnić się, że następujące narzędzia są zainstalowane na komputerze:

\begin{lstlisting}[language=bash]
# Sprawdzenie wersji Azure CLI
az --version  # Wymagana wersja >= 2.40.0

# Sprawdzenie wersji Terraform
terraform --version  # Wymagana wersja >= 1.0.0

# Sprawdzenie wersji Python
python --version  # Wymagana wersja >= 3.8

# Azure Functions Core Tools
func --version  # Wymagana wersja >= 4.0.0
\end{lstlisting}

\subsection{Logowanie do Azure}

Logowanie do Azure CLI jest konieczne, aby Terraform mógł zarządzać zasobami w Azure:

\begin{lstlisting}[language=bash]
# Logowanie do Azure
az login

# Lista dostępnych subskrypcji
az account list --output table

# Wybór konkretnej subskrypcji (opcjonalnie)
az account set --subscription "Nazwa_Subskrypcji_Azure"
\end{lstlisting}

\section{Definicja infrastruktury w Terraform}

\subsection{Struktura projektu}

Tworzymy następującą strukturę plików dla naszej konfiguracji Terraform:

\begin{lstlisting}[language=bash]
mkdir -p azure-functions-lab
cd azure-functions-lab
touch main.tf variables.tf outputs.tf terraform.tfvars
\end{lstlisting}

\subsection{Plik zmiennych (variables.tf)}

Plik ten definiuje zmienne używane w głównym pliku konfiguracyjnym, co pozwala na łatwą parametryzację wdrożenia:

\begin{lstlisting}[language=HCL]
variable "resource_group_name" {
  description = "Nazwa grupy zasobów"
  type        = string
  default     = "lab-functions-rg"
}

variable "location" {
  description = "Lokalizacja zasobów Azure"
  type        = string
  default     = "West Europe"
}

variable "storage_account_name" {
  description = "Nazwa konta storage (musi być unikalna globalnie)"
  type        = string
}

variable "app_insights_name" {
  description = "Nazwa Application Insights"
  type        = string
  default     = "lab-functions-insights"
}

variable "app_service_plan_name" {
  description = "Nazwa App Service Plan"
  type        = string
  default     = "lab-functions-plan"
}

variable "function_app_name" {
  description = "Nazwa Function App (musi być unikalna globalnie)"
  type        = string
}
\end{lstlisting}

\subsection{Główny plik konfiguracyjny (main.tf)}

W tym pliku definiujemy wszystkie zasoby Azure, które chcemy utworzyć dla naszej aplikacji Functions. Zwróć uwagę, że tworzymy aplikację dla środowiska Linuxowego z runtime'em Python:

\begin{lstlisting}[language=HCL]
provider "azurerm" {
  features {}
}

# Grupa zasobów
resource "azurerm_resource_group" "rg" {
  name     = var.resource_group_name
  location = var.location
}

# Storage Account wymagany dla Azure Functions
resource "azurerm_storage_account" "storage" {
  name                     = var.storage_account_name
  resource_group_name      = azurerm_resource_group.rg.name
  location                 = azurerm_resource_group.rg.location
  account_tier             = "Standard"
  account_replication_type = "LRS"
}

# Kontener dla przechowywania kodu funkcji
resource "azurerm_storage_container" "deployments" {
  name                  = "function-releases"
  storage_account_name  = azurerm_storage_account.storage.name
  container_access_type = "private"
}

# Storage dla przechowywania plików dla Blob Trigger
resource "azurerm_storage_container" "blob_trigger" {
  name                  = "samples-workitems"
  storage_account_name  = azurerm_storage_account.storage.name
  container_access_type = "private"
}

# Application Insights do monitorowania
resource "azurerm_application_insights" "insights" {
  name                = var.app_insights_name
  location            = azurerm_resource_group.rg.location
  resource_group_name = azurerm_resource_group.rg.name
  application_type    = "web"
}

# Plan usługi dla Azure Functions
resource "azurerm_service_plan" "app_service_plan" {
  name                = var.app_service_plan_name
  resource_group_name = azurerm_resource_group.rg.name
  location            = azurerm_resource_group.rg.location
  os_type             = "Linux"
  sku_name            = "Y1" # Plan Consumption
}

# Azure Linux Function App z runtime Python
resource "azurerm_linux_function_app" "function_app" {
  name                = var.function_app_name
  resource_group_name = azurerm_resource_group.rg.name
  location            = azurerm_resource_group.rg.location

  storage_account_name       = azurerm_storage_account.storage.name
  storage_account_access_key = azurerm_storage_account.storage.primary_access_key
  service_plan_id            = azurerm_service_plan.app_service_plan.id
  
  site_config {
    application_stack {
      python_version = "3.9"
    }
  }

  app_settings = {
    "APPINSIGHTS_INSTRUMENTATIONKEY" = azurerm_application_insights.insights.instrumentation_key
    "FUNCTIONS_WORKER_RUNTIME"       = "python"
    "WEBSITE_RUN_FROM_PACKAGE"       = "1"
    "AzureWebJobsStorage"            = azurerm_storage_account.storage.primary_connection_string
  }
}
\end{lstlisting}

\subsection{Plik wyjść (outputs.tf)}

Ten plik definiuje wartości, które chcemy wyświetlić po zakończeniu wdrażania infrastruktury:

\begin{lstlisting}[language=HCL]
output "function_app_name" {
  value = azurerm_linux_function_app.function_app.name
}

output "function_app_default_hostname" {
  value = azurerm_linux_function_app.function_app.default_hostname
}

output "function_app_id" {
  value = azurerm_linux_function_app.function_app.id
}

output "storage_account_name" {
  value = azurerm_storage_account.storage.name
}

output "storage_connection_string" {
  value     = azurerm_storage_account.storage.primary_connection_string
  sensitive = true
}
\end{lstlisting}

\subsection{Plik terraform.tfvars}

Ten plik zawiera faktyczne wartości zmiennych dla naszego wdrożenia:

\begin{lstlisting}[language=HCL]
storage_account_name = "labfunctstorage<ID>"
function_app_name    = "lab-functions-app-<ID>"
\end{lstlisting}

Uwaga: Zastąp \texttt{<ID>} swoim unikalnym identyfikatorem, aby zapewnić globalną unikalność nazw w Azure.

\section{Wdrażanie infrastruktury}

\subsection{Inicjalizacja Terraform}

Inicjalizacja pobiera wymagane providery i moduły:

\begin{lstlisting}[language=bash]
terraform init
\end{lstlisting}

Podczas inicjalizacji Terraform:
\begin{itemize}
    \item Pobiera wymagane providery (w tym przypadku provider azurerm)
    \item Inicjalizuje lokalny katalog .terraform
    \item Przygotowuje środowisko do wdrożenia
\end{itemize}

\subsection{Generowanie planu wdrożenia}

Sprawdzamy, jakie zasoby zostaną utworzone:

\begin{lstlisting}[language=bash]
terraform plan
\end{lstlisting}

Podczas wykonywania planu Terraform:
\begin{itemize}
    \item Analizuje istniejący stan infrastruktury
    \item Porównuje go z zadeklarowanym stanem w plikach konfiguracyjnych
    \item Generuje plan działań (dodanie, zmiana lub usunięcie zasobów)
\end{itemize}

\subsection{Wdrożenie zasobów}

Tworzymy zasoby w Azure zgodnie z konfiguracją:

\begin{lstlisting}[language=bash]
terraform apply
\end{lstlisting}

Po potwierdzeniu (wpisanie 'yes'), Terraform:
\begin{itemize}
    \item Tworzy grupę zasobów
    \item Tworzy konto Storage Account
    \item Konfiguruje kontenery Blob Storage
    \item Tworzy Application Insights
    \item Tworzy plan App Service dla Linux
    \item Wdraża aplikację Function App z runtime Python
\end{itemize}

\section{Tworzenie aplikacji Azure Functions w Pythonie}

\subsection{Inicjalizacja projektu Azure Functions}

Tworzymy lokalny projekt funkcji:

\begin{lstlisting}[language=bash]
mkdir -p functions-code
cd functions-code

# Inicjalizacja projektu Python dla Azure Functions
func init --worker-runtime python
\end{lstlisting}

Ta komenda tworzy:
\begin{itemize}
    \item Plik konfiguracyjny \texttt{host.json}
    \item Plik \texttt{local.settings.json} z ustawieniami lokalnymi
    \item Katalog \texttt{.venv} dla wirtualnego środowiska Python
    \item Plik \texttt{requirements.txt} do zarządzania zależnościami
\end{itemize}

\subsection{Endpoint 1: HTTP Trigger}

Tworzymy pierwszą funkcję HTTP:

\begin{lstlisting}[language=bash]
func new --name HttpTriggerFunction --template "HTTP trigger" --authlevel "function"
\end{lstlisting}

Ta komenda generuje:
\begin{itemize}
    \item Katalog \texttt{HttpTriggerFunction}
    \item Plik \texttt{\_\_init\_\_.py} z kodem funkcji
    \item Plik \texttt{function.json} z konfiguracją bindingów
\end{itemize}

Edytujemy plik \texttt{HttpTriggerFunction/\_\_init\_\_.py}:

\begin{lstlisting}[language=Python]
import logging
import azure.functions as func
import json
import datetime

def main(req: func.HttpRequest) -> func.HttpResponse:
    logging.info('Python HTTP trigger function processed a request.')

    name = req.params.get('name')
    if not name:
        try:
            req_body = req.get_json()
        except ValueError:
            pass
        else:
            name = req_body.get('name')

    if name:
        current_time = datetime.datetime.now().strftime("%Y-%m-%d %H:%M:%S")
        response_data = {
            "message": f"Witaj, {name}!",
            "timestamp": current_time,
            "endpoint": "HTTP Trigger"
        }
        return func.HttpResponse(
            json.dumps(response_data),
            mimetype="application/json",
            status_code=200
        )
    else:
        return func.HttpResponse(
            json.dumps({"error": "Proszę podać parametr 'name' w query string lub w body requestu"}),
            mimetype="application/json",
            status_code=400
        )
\end{lstlisting}

\subsection{Endpoint 2: Timer Trigger}

Tworzymy funkcję uruchamianą według harmonogramu:

\begin{lstlisting}[language=bash]
func new --name TimerTriggerFunction --template "Timer trigger"
\end{lstlisting}

Ta komenda generuje:
\begin{itemize}
    \item Katalog \texttt{TimerTriggerFunction}
    \item Plik \texttt{\_\_init\_\_.py} z kodem funkcji
    \item Plik \texttt{function.json} z konfiguracją harmonogramu (domyślnie co 5 minut)
\end{itemize}

Edytujemy plik \texttt{TimerTriggerFunction/\_\_init\_\_.py}:

\begin{lstlisting}[language=Python]
import datetime
import logging
import azure.functions as func
import os
import json
from azure.storage.blob import BlobServiceClient

def main(mytimer: func.TimerRequest) -> None:
    utc_timestamp = datetime.datetime.utcnow().replace(
        tzinfo=datetime.timezone.utc).isoformat()

    if mytimer.past_due:
        logging.info('Timer jest opóźniony!')

    logging.info('Python timer trigger function uruchomiona o %s', utc_timestamp)
    
    # Tworzymy raport z uruchomienia
    report_data = {
        "execution_time": utc_timestamp,
        "status": "success",
        "message": "Planowe uruchomienie timera"
    }
    
    # Zapisujemy raport do blob storage (opcjonalne)
    try:
        # Połączenie z storage
        connect_str = os.environ["AzureWebJobsStorage"]
        blob_service_client = BlobServiceClient.from_connection_string(connect_str)
        
        # Tworzenie nazwy pliku z timestampem
        report_filename = f"timer-report-{datetime.datetime.utcnow().strftime('%Y%m%d-%H%M%S')}.json"
        
        # Uzyskanie klienta kontenera
        container_client = blob_service_client.get_container_client("samples-workitems")
        
        # Zapis danych do bloba
        blob_client = container_client.get_blob_client(report_filename)
        blob_client.upload_blob(json.dumps(report_data))
        
        logging.info(f"Raport zapisany do blob storage: {report_filename}")
    except Exception as e:
        logging.error(f"Błąd podczas zapisywania raportu: {str(e)}")
\end{lstlisting}

\subsection{Endpoint 3: Blob Trigger}

Tworzymy funkcję aktywowaną przez dodanie/modyfikację pliku w Blob Storage:

\begin{lstlisting}[language=bash]
func new --name BlobTriggerFunction --template "Blob trigger"
\end{lstlisting}

Ta komenda generuje:
\begin{itemize}
    \item Katalog \texttt{BlobTriggerFunction}
    \item Plik \texttt{\_\_init\_\_.py} z kodem funkcji
    \item Plik \texttt{function.json} z konfiguracją bindingu do Blob Storage
\end{itemize}

Edytujemy plik \texttt{BlobTriggerFunction/function.json}:

\begin{lstlisting}[language=json]
{
  "scriptFile": "__init__.py",
  "bindings": [
    {
      "name": "myblob",
      "type": "blobTrigger",
      "direction": "in",
      "path": "samples-workitems/{name}",
      "connection": "AzureWebJobsStorage"
    }
  ]
}
\end{lstlisting}

Edytujemy plik \texttt{BlobTriggerFunction/\_\_init\_\_.py}:

\begin{lstlisting}[language=Python]
import logging
import azure.functions as func
import json
import datetime

def main(myblob: func.InputStream):
    logging.info(f"Python Blob trigger function przetworzył blob \n"
                 f"Nazwa: {myblob.name}\n"
                 f"Rozmiar: {myblob.length} bajtów")
    
    try:
        # Próba odczytu zawartości bloba jako JSON
        content = myblob.read()
        content_str = content.decode('utf-8')
        
        # Parsowanie JSON
        try:
            data = json.loads(content_str)
            logging.info(f"Przetworzono plik JSON: {data}")
            
            # Tutaj możesz dodać logikę przetwarzania danych
            # np. analizę raportów wygenerowanych przez TimerTrigger
            
        except json.JSONDecodeError:
            logging.warning(f"Plik nie jest prawidłowym JSON: {content_str[:100]}...")
            
    except Exception as e:
        logging.error(f"Wystąpił błąd podczas przetwarzania pliku: {str(e)}")
\end{lstlisting}

\subsection{Aktualizacja zależności}

Dodajemy wymagane pakiety do pliku \texttt{requirements.txt}:

\begin{lstlisting}
azure-functions
azure-storage-blob
\end{lstlisting}

Instalujemy zależności:

\begin{lstlisting}[language=bash]
# Aktywacja wirtualnego środowiska (jeśli nie jest aktywne)
source .venv/bin/activate  # Na Linuxie/macOS
# lub
.venv\Scripts\activate     # Na Windows

# Instalacja zależności
pip install -r requirements.txt
\end{lstlisting}

\section{Wdrażanie funkcji do Azure}

\subsection{Lokalne testowanie (opcjonalne)}

Przed wdrożeniem możemy przetestować funkcje lokalnie:

\begin{lstlisting}[language=bash]
func start
\end{lstlisting}

\subsection{Publikowanie funkcji do Azure}

Wdrażamy nasze funkcje do utworzonej wcześniej infrastruktury:

\begin{lstlisting}[language=bash]
# Pakowanie i publikowanie funkcji
func azure functionapp publish <nazwa_funkcji>
\end{lstlisting}

Gdzie \texttt{<nazwa\_funkcji>} to wartość wyjściowa \texttt{function\_app\_name} z Terraform (lub wartość z pliku terraform.tfvars).

Podczas publikacji:
\begin{itemize}
    \item Kod jest pakowany do formatu zip
    \item Plik zip jest przesyłany do konta storage
    \item Function App jest konfigurowany do uruchomienia kodu z paczki
    \item Ustawienia aplikacji są aktualizowane
\end{itemize}

\section{Testowanie funkcji w Azure}

\subsection{Testowanie HTTP Trigger}

Pobierz URL funkcji z portalu Azure lub za pomocą Azure CLI:

\begin{lstlisting}[language=bash]
# Uzyskanie klucza funkcji
FUNCTION_KEY=$(az functionapp keys list -g <nazwa_grupy_zasobów> -n <nazwa_funkcji> --query "functionKeys.default" -o tsv)

# Zapytanie HTTP
curl "https://<nazwa_funkcji>.azurewebsites.net/api/httptriggerfunction?name=Student&code=$FUNCTION_KEY"
\end{lstlisting}

Oczekiwana odpowiedź:
\begin{lstlisting}[language=json]
{
  "message": "Witaj, Student!",
  "timestamp": "2025-03-16 20:45:30",
  "endpoint": "HTTP Trigger"
}
\end{lstlisting}

\subsection{Testowanie Timer Trigger}

Timer trigger uruchamia się automatycznie zgodnie z harmonogramem (domyślnie co 5 minut).

Sprawdzamy logi funkcji w portalu Azure lub za pomocą Azure CLI:

\begin{lstlisting}[language=bash]
az functionapp log tail --name <nazwa_funkcji> --resource-group <nazwa_grupy_zasobów>
\end{lstlisting}

W logach powinniśmy zobaczyć informacje o uruchomieniu funkcji timer i zapisie raportu do Blob Storage.

\subsection{Testowanie Blob Trigger}

Aby przetestować Blob Trigger, musimy wgrać plik do kontenera \texttt{samples-workitems}:

\begin{lstlisting}[language=bash]
# Tworzenie testowego pliku JSON
echo '{"test": "data", "timestamp": "'$(date -u +"%Y-%m-%dT%H:%M:%SZ")'"}' > test-file.json

# Wgranie pliku do blob storage
az storage blob upload --account-name <nazwa_konta_storage> --container-name samples-workitems --name test-blob.json --file test-file.json --auth-mode login
\end{lstlisting}

Po wgraniu pliku, Blob Trigger powinien się uruchomić i przetworzyć plik. Efekty można sprawdzić w logach.

\section{Monitorowanie aplikacji}

\subsection{Application Insights}

Przejdź do Azure Portal i znajdź zasób Application Insights utworzony przez Terraform.

Kluczowe sekcje do przejrzenia:
\begin{itemize}
    \item Live Metrics - monitorowanie w czasie rzeczywistym
    \item Failures - błędy aplikacji
    \item Performance - wydajność
    \item Logs - szczegółowe logi
\end{itemize}

Przykładowe kwerendy Kusto w Log Analytics:

\begin{lstlisting}[language=SQL]
// Wykonania funkcji HTTP
requests
| where cloud_RoleName startswith "lab-functions"
| where name contains "HttpTriggerFunction"
| project timestamp, name, success, resultCode, duration, operation_Id

// Wykonania wszystkich funkcji
traces
| where cloud_RoleName startswith "lab-functions"
| project timestamp, message, severityLevel, operation_Id
| order by timestamp desc
\end{lstlisting}

\section{Czyszczenie zasobów}

Po zakończeniu laboratorium, należy usunąć wszystkie utworzone zasoby, aby uniknąć niepotrzebnych kosztów:

\begin{lstlisting}[language=bash]
terraform destroy
\end{lstlisting}

Terraform poprosi o potwierdzenie operacji. Po wpisaniu 'yes', wszystkie utworzone zasoby zostaną usunięte w odpowiedniej kolejności.

\section{Zadania do wykonania dla studentów}

\begin{enumerate}
    \item Modyfikacja funkcji HTTP:
    \begin{itemize}
        \item Dodanie obsługi dodatkowych parametrów (np. age, location)
        \item Implementacja różnych metod HTTP (GET, POST, PUT)
        \item Dodanie walidacji danych wejściowych
    \end{itemize}
    
    \item Modyfikacja funkcji Timer:
    \begin{itemize}
        \item Zmiana harmonogramu na własny (CRON expression)
        \item Implementacja dodatkowej logiki biznesowej
        \item Zapisywanie raportów w strukturyzowanym formacie
    \end{itemize}
    
    \item Modyfikacja funkcji Blob Trigger:
    \begin{itemize}
        \item Przetwarzanie różnych typów plików (np. JSON, CSV)
        \item Implementacja analizy zawartości plików
        \item Generowanie powiadomień o przetworzonych plikach
    \end{itemize}
\end{enumerate}

\section{Wnioski}

W ramach tego laboratorium studenci nauczyli się:

\begin{itemize}
    \item Definiować infrastrukturę jako kod przy użyciu Terraform
    \item Tworzyć i wdrażać aplikacje serverless w Azure Functions
    \item Implementować różne typy triggerów funkcji w Pythonie
    \item Integrować funkcje z usługami Azure, takimi jak Blob Storage
    \item Monitorować działanie aplikacji przy użyciu Application Insights
\end{itemize}

Zdobyte umiejętności są niezbędne w nowoczesnym podejściu do projektowania systemów rozproszonych, gdzie elastyczność, skalowalność i łatwość zarządzania są kluczowymi cechami.

\end{document}

Źródła
