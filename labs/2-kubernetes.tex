\documentclass{article}
\usepackage[utf8]{inputenc}
\usepackage{listings}
\usepackage{color}
\usepackage{url}
\setlength{\parindent}{0pt}

% YAML code color settings for listings package
\definecolor{gray}{rgb}{0.5,0.5,0.5}
\definecolor{orange}{rgb}{1,0.5,0}
\lstset{
    language=bash,
    basicstyle=\ttfamily,
    keywordstyle=\color{blue},
    commentstyle=\color{gray},
    stringstyle=\color{orange},
    breaklines=true,
    breakatwhitespace=true,
    showstringspaces=true
}
\title{Zarządzanie Systemami Rozproszonymi\\Laboratoria z Kubernetes}
\author{mgr inż. Jakub Woźniak}
\date{}

\begin{document}

\maketitle

\section{Wprowadzenie}
Kubernetes to platforma do automatyzacji zarządzania kontenerami. W ramach tych laboratoriów studenci będą mieli okazję pracować z Minikube – lokalną instancją Kubernetes – aby poznać podstawowe zasoby systemu, takie jak Pody, Deploymenty, PersistentVolume, ConfigMap i Secret.

\subsection{Instalacja Minikube}
Minikube to narzędzie, które umożliwia uruchomienie klastra Kubernetes lokalnie na komputerze. Aby zainstalować Minikube na openSUSE, użyj poniższych komend:
\begin{lstlisting}
sudo zypper install -y virtualbox
curl -LO https://storage.googleapis.com/minikube/releases/latest/minikube-linux-amd64
sudo install minikube-linux-amd64 /usr/local/bin/minikube
minikube start --driver=virtualbox
\end{lstlisting}

Po instalacji warto sprawdzić, czy klaster Kubernetes został poprawnie uruchomiony, za pomocą:
\begin{lstlisting}
kubectl get nodes
\end{lstlisting}
Komenda ta wyświetla listę węzłów w klastrze. Jeśli wszystko działa poprawnie, powinien być widoczny węzeł \texttt{Ready}.

\section{Podstawy Kubernetes}

\subsection{Pod – Co to jest i dlaczego jest ważny?}
Pod to najmniejsza jednostka zarządzania w Kubernetes, która uruchamia kontenery. Pody grupują jeden lub więcej kontenerów, współdzieląc te same zasoby, takie jak sieć i woluminy. Każdy Pod działa z własnym adresem IP, a Kubernetes zarządza jego cyklem życia.

Przykładowy YAML do uruchomienia poda Nginx:
\begin{lstlisting}
apiVersion: v1
kind: Pod
metadata:
  name: nginx
spec:
  containers:
  - name: nginx
    image: nginx:latest
    ports:
    - containerPort: 80
\end{lstlisting}

\textbf{Wyjaśnienie elementów YAML:}
\begin{itemize}
  \item \texttt{apiVersion: v1} – Wersja API Kubernetes używana do definiowania zasobu. Wersja v1 to stabilna wersja API dla podstawowych zasobów.
  \item \texttt{kind: Pod} – Określa rodzaj zasobu. W tym przypadku jest to Pod, który uruchamia jeden lub więcej kontenerów.
  \item \texttt{metadata:} – Sekcja, która zawiera metadane, takie jak nazwa zasobu. W tym przypadku Pod nosi nazwę \texttt{nginx}.
  \item \texttt{spec:} – Specyfikacja opisuje, jakie kontenery zostaną uruchomione w Podzie. Zawiera listę kontenerów, w tym ich obraz (\texttt{nginx:latest}) oraz porty (\texttt{80}).
\end{itemize}

\textbf{Kroki wdrożenia i sprawdzenia:}
\begin{itemize}
  \item Uruchom pod:
  \begin{lstlisting}
  kubectl apply -f nginx-pod.yaml
  \end{lstlisting}
  \item Sprawdź status poda:
  \begin{lstlisting}
  kubectl get pods
  \end{lstlisting}
  Powinieneś zobaczyć pod o nazwie \texttt{nginx} w statusie \texttt{Running}.
  \item Aby zobaczyć logi kontenera:
  \begin{lstlisting}
  kubectl logs nginx
  \end{lstlisting}
\end{itemize}

\subsection{Deployment – Dlaczego go potrzebujemy?}
Deployment to zasób Kubernetes, który automatycznie zarządza Podami, zapewniając, że określona liczba replik (kopii) aplikacji jest zawsze uruchomiona. Deployment umożliwia również bezpieczne wdrażanie nowych wersji aplikacji poprzez tzw. rolling update. Kubernetes monitoruje stan Podów i automatycznie odtwarza je w przypadku awarii.

Przykład YAML dla Deploymentu Nginx:
\begin{lstlisting}
apiVersion: apps/v1
kind: Deployment
metadata:
  name: nginx-deployment
spec:
  replicas: 3
  selector:
    matchLabels:
      app: nginx
  template:
    metadata:
      labels:
        app: nginx
    spec:
      containers:
      - name: nginx
        image: nginx:latest
        ports:
        - containerPort: 80
\end{lstlisting}

\textbf{Wyjaśnienie elementów YAML:}
\begin{itemize}
  \item \texttt{replicas: 3} – Definiuje liczbę replik podów, które Kubernetes ma utrzymywać.
  \item \texttt{selector: matchLabels:} – Wybiera Pody na podstawie etykiety (\texttt{app: nginx}).
  \item \texttt{template:} – Szablon dla nowych Podów tworzonych przez ten Deployment. Zawiera specyfikację, która definiuje kontenery i ich ustawienia.
\end{itemize}

\textbf{Kroki wdrożenia i sprawdzenia:}
\begin{itemize}
  \item Uruchom Deployment:
  \begin{lstlisting}
  kubectl apply -f nginx-deployment.yaml
  \end{lstlisting}
  \item Sprawdź status Deploymentu:
  \begin{lstlisting}
  kubectl get deployments
  kubectl get pods
  \end{lstlisting}
  Powinieneś zobaczyć trzy pody \texttt{nginx} uruchomione i gotowe.
  \item Usuń pod, aby zobaczyć, jak Kubernetes automatycznie go odtworzy:
  \begin{lstlisting}
  kubectl delete pod <nazwa-poda>
  \end{lstlisting}
\end{itemize}

\subsection{Service – Umożliwienie dostępu do aplikacji}
Service to zasób Kubernetes, który umożliwia komunikację między Podami oraz udostępnienie aplikacji na zewnątrz klastra. Service zapewnia stabilny adres IP i port, nawet jeśli Pody są usuwane i tworzone na nowo.

Przykład YAML dla Service typu NodePort:
\begin{lstlisting}
apiVersion: v1
kind: Service
metadata:
  name: nginx-service
spec:
  type: NodePort
  selector:
    app: nginx
  ports:
  - protocol: TCP
    port: 80
    targetPort: 80
    nodePort: 30007
\end{lstlisting}

\textbf{Wyjaśnienie elementów YAML:}
\begin{itemize}
  \item \texttt{type: NodePort} – Umożliwia dostęp do aplikacji spoza klastra na określonym porcie (\texttt{30007}).
  \item \texttt{selector:} – Łączy Service z Podami oznaczonymi etykietą \texttt{app: nginx}.
  \item \texttt{ports:} – Definiuje porty, przez które aplikacja będzie dostępna (port 80 w Podzie, a port 30007 na nodzie).
\end{itemize}

\textbf{Kroki wdrożenia i sprawdzenia:}
\begin{itemize}
  \item Uruchom Service:
  \begin{lstlisting}
  kubectl apply -f nginx-service.yaml
  \end{lstlisting}
  \item Sprawdź, na jakim porcie usługa jest dostępna:
  \begin{lstlisting}
  kubectl get svc
  \end{lstlisting}
    \item Aby uzyskać dostęp do usługi, użyj Minikube do wygenerowania URL:
  \begin{lstlisting}
  minikube service nginx-service --url
  \end{lstlisting}
\end{itemize}

\section{PersistentVolume i PersistentVolumeClaim – Trwałe przechowywanie danych}

PersistentVolume (PV) to zasób klastra, który reprezentuje fizyczną przestrzeń dyskową do trwałego przechowywania danych. PersistentVolumeClaim (PVC) to żądanie aplikacji do korzystania z zasobu PV. Dzięki PV i PVC aplikacje mogą przechowywać dane, które przetrwają ponowne uruchomienie Podów.

Przykład YAML dla PersistentVolume (PV):
\begin{lstlisting}
apiVersion: v1
kind: PersistentVolume
metadata:
  name: mysql-pv
spec:
  capacity:
    storage: 1Gi
  accessModes:
    - ReadWriteOnce
  hostPath:
    path: "/mnt/data"
\end{lstlisting}

Przykład YAML dla PersistentVolumeClaim (PVC):
\begin{lstlisting}
apiVersion: v1
kind: PersistentVolumeClaim
metadata:
  name: mysql-pvc
spec:
  accessModes:
    - ReadWriteOnce
  resources:
    requests:
      storage: 1Gi
\end{lstlisting}

\textbf{Kroki wdrożenia i sprawdzenia:}
\begin{itemize}
  \item Uruchom PersistentVolume i PersistentVolumeClaim:
  \begin{lstlisting}
  kubectl apply -f pv.yaml
  kubectl apply -f pvc.yaml
  \end{lstlisting}
  \item Sprawdź status PV i PVC:
  \begin{lstlisting}
  kubectl get pv
  kubectl get pvc
  \end{lstlisting}
  Pomyślne przyznanie zasobów oznacza, że PVC zostało poprawnie podłączone do PV.
\end{itemize}

\section{ConfigMap i Secret – Przechowywanie konfiguracji i wrażliwych danych}

\subsection{ConfigMap} 
W tym zadaniu studenci będą korzystać z ConfigMap, aby zmienić stronę główną serwera NGINX. Skonfigurujemy Deployment, aby NGINX korzystał z pliku \texttt{index.html} przechowywanego w ConfigMap, oraz udostępnimy aplikację przy użyciu Service typu NodePort.

\textbf{ConfigMap YAML:}
\begin{lstlisting}
apiVersion: v1
kind: ConfigMap
metadata:
  name: nginx-config
data:
  index.html: |
    <html>
    <head><title>NGINX</title></head>
    <body><h1>Witamy na serwerze NGINX</h1></body>
    </html>
\end{lstlisting}

\textbf{Deployment YAML:}
\begin{lstlisting}
apiVersion: apps/v1
kind: Deployment
metadata:
  name: nginx-deployment
spec:
  replicas: 2
  selector:
    matchLabels:
      app: nginx
  template:
    metadata:
      labels:
        app: nginx
    spec:
      containers:
      - name: nginx
        image: nginx:latest
        volumeMounts:
        - name: html
          mountPath: /usr/share/nginx/html
      volumes:
      - name: html
        configMap:
          name: nginx-config
\end{lstlisting}

\textbf{Wyjaśnienie nowych elementów YAML:}
\begin{itemize}
  \item \texttt{volumeMounts:} – Określa, gdzie w kontenerze ma być zamontowany wolumin z danymi. W tym przypadku plik \texttt{index.html} jest zamontowany w lokalizacji \texttt{/usr/share/nginx/html}, czyli w miejscu, gdzie NGINX przechowuje pliki stron.
  \item \texttt{volumes: configMap:} – Wolumin ConfigMap, który umożliwia montowanie danych z ConfigMap w kontenerze. W tym przypadku wolumin montuje \texttt{nginx-config} i używa go do zaktualizowania strony głównej NGINX.
\end{itemize}

\textbf{Service YAML:}
\begin{lstlisting}
apiVersion: v1
kind: Service
metadata:
  name: nginx-service
spec:
  type: NodePort
  selector:
    app: nginx
  ports:
  - protocol: TCP
    port: 80
    targetPort: 80
    nodePort: 30007
\end{lstlisting}

\textbf{Wyjaśnienie nowych elementów YAML:}
\begin{itemize}
  \item \texttt{type: NodePort} – Umożliwia dostęp do aplikacji spoza klastra poprzez przypisanie portu (w tym przypadku \texttt{30007}) na wszystkich węzłach klastra.
  \item \texttt{selector:} – Wybiera pody na podstawie etykiety (\texttt{app: nginx}), co pozwala na kierowanie ruchu do odpowiednich instancji NGINX uruchomionych przez Deployment.
\end{itemize}

\textbf{Kroki wdrożenia i sprawdzenia:}
\begin{itemize}
  \item Uruchom ConfigMap:
  \begin{lstlisting}
  kubectl apply -f nginx-config.yaml
  \end{lstlisting}
  \item Uruchom Deployment:
  \begin{lstlisting}
  kubectl apply -f nginx-deployment.yaml
  \end{lstlisting}
  \item Uruchom Service:
  \begin{lstlisting}
  kubectl apply -f nginx-service.yaml
  \end{lstlisting}
  \item Sprawdź, na jakim porcie usługa jest dostępna:
  \begin{lstlisting}
  kubectl get svc
  \end{lstlisting}
  \item Uzyskaj dostęp do NGINX, używając Minikube do wygenerowania URL:
  \begin{lstlisting}
  minikube service nginx-service --url
  \end{lstlisting}
  Otwórz stronę w przeglądarce i sprawdź, czy wyświetla się zmodyfikowana strona główna z ConfigMap.
\end{itemize}

\subsection{Secret}

Secret służy do bezpiecznego przechowywania wrażliwych informacji, takich jak hasła i klucze API. W tym zadaniu użyjemy Secret, aby przekazać klucz API do aplikacji, która będzie pobierać dane z serwisu \url{https://jsonplaceholder.typicode.com}.

\textbf{Secret YAML:}
\begin{lstlisting}
apiVersion: v1
kind: Secret
metadata:
  name: api-secret
data:
  api-key: c2VjcmV0a2V5
\end{lstlisting}

\textbf{Wyjaśnienie nowych elementów YAML:}
\begin{itemize}
  \item \texttt{kind: Secret} – Typ zasobu, który umożliwia bezpieczne przechowywanie wrażliwych informacji, takich jak hasła czy klucze API. Dane w \texttt{Secret} są zakodowane w formacie base64.
  \item \texttt{data: api-key} – Zawiera klucz API zakodowany w formacie base64. W tym przykładzie rzeczywista wartość klucza to \texttt{secretkey}.
\end{itemize}

Następnie aplikacja będzie korzystać z tego klucza API, aby pobierać dane JSON z \url{https://jsonplaceholder.typicode.com}.

\textbf{Pod YAML:}
\begin{lstlisting}
apiVersion: v1
kind: Pod
metadata:
  name: json-app
spec:
  containers:
  - name: json-app
    image: ubuntu:24.04
    env:
    - name: API_KEY
      valueFrom:
        secretKeyRef:
          name: api-secret
          key: api-key
    command: ["sh", "-c", "curl -H 'Authorization: Bearer $API_KEY' https://jsonplaceholder.typicode.com/posts"]
\end{lstlisting}

\textbf{Wyjaśnienie nowych elementów YAML:}
\begin{itemize}
  \item \texttt{env:} – Używane do definiowania zmiennych środowiskowych w kontenerze. W tym przypadku klucz API jest przekazywany jako zmienna \texttt{API\_KEY}.
  \item \texttt{valueFrom: secretKeyRef:} – Mechanizm pobierania wartości z \texttt{Secret}. W tym przypadku używamy klucza o nazwie \texttt{api-key} z zasobu \texttt{api-secret}.
  \item \texttt{command:} – Używane do uruchamiania polecenia \texttt{curl}, które pobiera dane JSON z serwisu \url{https://jsonplaceholder.typicode.com}, używając klucza API jako nagłówka autoryzacyjnego.
\end{itemize}

\textbf{Kroki wdrożenia i sprawdzenia:}
\begin{itemize}
  \item Uruchom Secret:
  \begin{lstlisting}
  kubectl apply -f api-secret.yaml
  \end{lstlisting}
  \item Uruchom Pod z aplikacją:
  \begin{lstlisting}
  kubectl apply -f json-app.yaml
  \end{lstlisting}
  \item Sprawdź logi Poda, aby zobaczyć dane pobrane z \url{https://jsonplaceholder.typicode.com/posts}:
  \begin{lstlisting}
  kubectl logs json-app
  \end{lstlisting}
\end{itemize}

\section{Zadanie główne: Wdrożenie Ghosta z MySQL}
\subsection{Opis zadania}
Celem zadania jest samodzielne wdrożenie aplikacji Ghost z bazą danych MySQL na Kubernetes, wykorzystując wiedzę zdobywaną podczas wcześniejszych ćwiczeń.

\subsection{Wymagania}
\begin{itemize}
  \item Aplikacja Ghost musi być wdrożona z użyciem Deploymentu.
  \item Dane MySQL muszą być przechowywane w PersistentVolume, aby przetrwały ponowne uruchomienie Podów.
  \item Hasło do bazy MySQL musi być bezpiecznie przechowywane w zasobie Secret.
  \item Aplikacja Ghost musi być dostępna na zewnątrz klastra przez Service typu NodePort lub LoadBalancer.
\end{itemize}

\subsection{Kryteria zaliczenia}
\begin{itemize}
  \item Aplikacja Ghost działa i jest dostępna przez przeglądarkę.
  \item Dane w bazie MySQL są trwałe i przetrwały ponowne uruchomienie Poda.
  \item Deploymenty Ghost i MySQL automatycznie odtwarzają Pody w przypadku awarii.
  \item Hasło do bazy danych jest poprawnie przechowywane w zasobie Secret.
  \item Zastosowany PersistentVolume umożliwia trwałe przechowywanie danych MySQL.
\end{itemize}

\subsection{Linki do dokumentacji}
\begin{itemize}
  \item Ghost: \url{https://ghost.org/docs/}
  \item MySQL: \url{https://dev.mysql.com/doc/}
  \item Kubernetes Secrets: \url{https://kubernetes.io/docs/concepts/configuration/secret/}
\end{itemize}
\section{Horizontal Pod Autoscaler (HPA)}

Na koniec, skonfigurujemy Horizontal Pod Autoscaler (HPA), który automatycznie skalibruje liczbę replik aplikacji Ghost w zależności od obciążenia CPU.

Przykład YAML dla HPA:
\begin{lstlisting}
apiVersion: autoscaling/v1
kind: HorizontalPodAutoscaler
metadata:
  name: ghost-hpa
spec:
  scaleTargetRef:
    apiVersion: apps/v1
    kind: Deployment
    name: ghost
  minReplicas: 1
  maxReplicas: 5
  targetCPUUtilizationPercentage: 50
\end{lstlisting}

Aby przetestować HPA, można użyć narzędzia \texttt{Apache Benchmark} (ab), które wygeneruje ruch na aplikację Ghost:
\begin{lstlisting}
ab -n 1000 -c 50 http://<ghost-service-url>/
\end{lstlisting}

\textbf{Weryfikacja działania:}
\begin{itemize}
  \item Sprawdź, czy liczba replik aplikacji wzrasta:
  \begin{lstlisting}
  kubectl get hpa
  kubectl get pods
  \end{lstlisting}
\end{itemize}

\end{document}

