\documentclass{article}
\usepackage{hyperref}
\usepackage{listings}
\usepackage{color}

\title{Laboratoria z Ansible - Scenariusz}
\author{Zarządzanie Systemami Rozproszonymi}

\begin{document}

\maketitle

\section*{Wprowadzenie}

\textbf{Cel laboratoriów}:  
Celem tych laboratoriów jest zapoznanie studentów z narzędziem Ansible oraz automatyzacją zarządzania konfiguracją. Studenci będą używać Ansible do instalacji oprogramowania, zarządzania plikami oraz konfiguracji systemów w sieci za pomocą playbooków YAML. Laboratoria kończą się zadaniem zbudowania klastra Redis z jednym węzłem Master i dwoma węzłami Slave.

\textbf{Wymagana wiedza}:  
Znajomość podstaw systemu Linux (komendy terminala) oraz podstawowa znajomość sieci (np. SSH, IP).

\textbf{Narzędzia}:  
- Maszyna kontrolna (np. openSUSE) z zainstalowanym Ansible.
- Trzy maszyny wirtualne w tej samej sieci, na których będą wykonywane zadania.
- Połączenie SSH między maszynami.

\textbf{Rezultat}:  
Studenci nauczą się używać Ansible do zarządzania wieloma maszynami jednocześnie za pomocą playbooków YAML. Efektem końcowym będzie działający klaster Redis.

\section{Krok 1: Przygotowanie środowiska}

\textbf{Instalacja Ansible na maszynie kontrolnej:}  
Na maszynie kontrolnej (np. openSUSE) należy zainstalować Ansible:

\begin{lstlisting}
sudo zypper install ansible
\end{lstlisting}

Po instalacji upewnij się, że Ansible działa:

\begin{lstlisting}
ansible --version
\end{lstlisting}

Link do dokumentacji Ansible:  
\url{https://docs.ansible.com/ansible/latest/installation_guide/intro_installation.html}

\textbf{Utworzenie pliku \texttt{hosts}}:  
W katalogu \texttt{/etc/ansible/} utwórz plik \texttt{hosts}, w którym wymienione będą maszyny docelowe:

\begin{lstlisting}
[myservers]
vm1 ansible_host=192.168.1.101
vm2 ansible_host=192.168.1.102
vm3 ansible_host=192.168.1.103
\end{lstlisting}

\section{Krok 2: Sprawdzenie połączenia}

\textbf{Polecenie \texttt{ping}}:  
Aby upewnić się, że Ansible może połączyć się z maszynami:

\begin{lstlisting}
ansible myservers -m ping
\end{lstlisting}

Połączone maszyny odpowiedzą "pong".  
Link do dokumentacji modułu \texttt{ping}:  
\url{https://docs.ansible.com/ansible/latest/collections/ansible/builtin/ping_module.html}

\section{Krok 3: Zbieranie faktów}

\textbf{Zbieranie faktów z maszyn:}

\begin{lstlisting}
ansible myservers -m gather_facts
\end{lstlisting}

Moduł \texttt{gather\_facts} dostarcza szczegółowych informacji o systemach (adresy IP, system operacyjny itd.).  

Link do dokumentacji:  
\url{https://docs.ansible.com/ansible/latest/collections/ansible/builtin/setup_module.html}

\section{Krok 4: Instalacja Nginx za pomocą playbooka}

\textbf{Playbook YAML do instalacji Nginx:}  
Od tego momentu studenci będą korzystać wyłącznie z playbooków w formacie YAML. Stwórz playbook o nazwie \texttt{nginx-install.yml}:

\begin{lstlisting}
---
- hosts: myservers
  become: yes
  tasks:
    - name: Install Nginx
      apt:
        name: nginx
        state: present
\end{lstlisting}

Wykonaj playbook:

\begin{lstlisting}
ansible-playbook nginx-install.yml
\end{lstlisting}

Link do dokumentacji modułu \texttt{apt}:  
\url{https://docs.ansible.com/ansible/latest/collections/ansible/builtin/apt_module.html}

\section{Krok 5: Modyfikacja strony głównej za pomocą playbooka}

\textbf{Tworzenie i kopiowanie pliku \texttt{index.html}}:  
Napisz playbook o nazwie \texttt{update-index.yml}, który skopiuje plik \texttt{index.html} do katalogu Nginx:

\begin{lstlisting}
---
- hosts: myservers
  become: yes
  tasks:
    - name: Copy index.html to all servers
      copy:
        src: /path/to/index.html
        dest: /var/www/html/index.html
\end{lstlisting}

Wykonaj playbook:

\begin{lstlisting}
ansible-playbook update-index.yml
\end{lstlisting}

Link do dokumentacji modułu \texttt{copy}:  
\url{https://docs.ansible.com/ansible/latest/collections/ansible/builtin/copy_module.html}

\section{Krok 6: Użycie szablonu do dynamicznej strony}

\textbf{Tworzenie dynamicznego szablonu strony głównej:}  
Napisz szablon \texttt{index.html.j2}, który będzie zawierał adres IP każdej maszyny:

\begin{lstlisting}
<html>
  <head><title>Ansible Lab</title></head>
  <body>
    <h1>This is {{ ansible_default_ipv4.address }}</h1>
  </body>
</html>
\end{lstlisting}

Następnie napisz playbook o nazwie \texttt{template-index.yml}, aby wdrożyć szablon:

\begin{lstlisting}
---
- hosts: myservers
  become: yes
  tasks:
    - name: Deploy template index.html
      template:
        src: /path/to/index.html.j2
        dest: /var/www/html/index.html
\end{lstlisting}

Wykonaj playbook:

\begin{lstlisting}
ansible-playbook template-index.yml
\end{lstlisting}

Link do dokumentacji modułu \texttt{template}:  
\url{https://docs.ansible.com/ansible/latest/collections/ansible/builtin/template_module.html}

\section{Krok 7: Instalacja Redis (Master-Slave)}

\textbf{Zadanie główne}:  
Zadanie to polega na skonfigurowaniu Redis w trybie Master-Slave z jedną maszyną Master i dwiema maszynami Slave.

\textbf{Polecenie}:  
Stwórz playbook, który zainstaluje Redis na wszystkich maszynach, a następnie skonfiguruj go, aby działał w trybie Master-Slave. Na jednej maszynie skonfiguruj Redis jako Master, a na dwóch pozostałych jako Slave.

\textbf{Weryfikacja}:  
Zaloguj się na maszynę Master i dodaj dane do Redis, a następnie sprawdź na maszynach Slave, czy dane zostały zreplikowane:

\begin{lstlisting}
redis-cli set mykey "Ansible Redis"
redis-cli get mykey
\end{lstlisting}

Dane muszą być replikowane na maszynach Slave.

\section*{Podsumowanie}
W trakcie tych laboratoriów studenci nauczyli się korzystać z Ansible, aby zarządzać konfiguracją na wielu maszynach przy pomocy playbooków YAML. Zakończyliśmy laboratoria, budując klaster Redis z konfiguracją Master-Slave i zweryfikowaliśmy replikację danych.
\end{document}

